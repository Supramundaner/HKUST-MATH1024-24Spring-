\documentclass{article}
\usepackage{amsfonts, amsmath, amsthm, bm, boondox-cal, color, enumitem
, float, fontspec, fullpage, geometry, graphicx, hyperref, mathabx, mathrsfs, setspace, subfigure, tikz-cd}
\title{\LaTeX{} Answer to Problem Set \#}
\author{AO Yuzhuo, JIN Zhibo, LYU Changlai\thanks{names listed in alphabetical order}}
\date{\today}

\everymath{\displaystyle}

\def\st{\text{ s.t. }}
\def\tif{\text{ if }}
\def\then{\text{ then }}
\def\R{\mathbb{R}}

\begin{document}
\maketitle

\section{}
\paragraph{(a)}~{}
\def\tempa{\cos(mx)\cos(nx)}
\def\tempb{\cos\left((m+n)x\right)}
\def\tempc{\cos\left((m-n)x\right)}
\def\d{\mathrm{d}}

\\\\When $m=n$, $\tempa=\cos(mx)^2=\frac{1}{2}+\frac{1}{2}\cos(2mx)$
\\Hence
$$\int_{0}^{2\pi}\tempa \d x=\int_{0}^{2\pi}\left(\frac{1}{2}+\frac{1}{2}\cos(2mx)\right)\d x=\left[\frac{x}{2}+\frac{\sin(2mx)}{4m}\right]_{0}^{2\pi}=\frac{2\pi}{2}+0-\frac{0}{2}-0=\pi$$
\\When $m\neq n$, by trigonometric equations:

$$ \cdot\tempa=\frac{1}{2}\cdot\left(\tempb+\tempc\right)$$
Since $m-n\neq 0$ in this case, 
\begin{flalign*}
    
	\;&\int_{0}^{2\pi}\tempa \d x\\
	=&\int_{0}^{2\pi} \frac{1}{2}\big(\tempb+\tempc\big)\d x\\
	=&\left[ \frac{1}{2}\cdot\left( \frac{\sin\left((m+n)x\right)}{m+n}+\frac{\sin\left((m-n)x\right)}{m-n}\right)\right]_{0}^{2\pi}\\
	=&0-0\\
	=&0
\end{flalign*}
In conclusion,
$$
\int_{0}^{2\pi}\tempa\d x=\left\{
\begin{aligned}
	\pi\quad &\text{if $m= n$}\\
	0  \quad &\text{if $m\neq n$}
\end{aligned}
\right.
$$
\paragraph{(b)}~{}
\def\ta{\sum_{i=1024}^{2024}\cos(ix)}
\def\tb{\sum_{i=1024}^{2024}(\cos(ix))^2}
\def\tc{\sum_{i=1024}^{2024}\sum_{j=1024}^{2024}\cos(ix)\cos(jx)}
\def\td{\sum_{i=1024}^{2023}\sum_{j=i+1}^{2024}\cos(ix)\cos(jx)}
\\\\We see that:
\begin{flalign*} 
	\;&\left( \ta\right)^2\\
	=&\tc\\
	=&\tb+2\td
\end{flalign*}
Moreover, from part (a) we know that:
 $$\int_{0}^{2\pi}\cos (ix) \cos (jx) \d x = 0 \text{ when } i\neq j \text{ and }   \int_{0}^{2\pi}(\cos (ix))^2 \d x=\pi.$$
\\Hence
\begin{flalign*}
	\;&\int_{0}^{2\pi}\left(\ta\right)^2 \d x\\
	=&\int_{0}^{2\pi}\left( \tb + \td\right) \d x\\
	=&1000\cdot \pi +0\\
	=&1000 \pi
\end{flalign*}
\\Hence
$$\int_{0}^{2\pi}\left(\ta\right)^2\d x=1000\pi$$

\section{}
\paragraph{(a)}Proof:\\
Suppose $u(x)=g(x)$, then $$\int_{a}^{b} g^2(x)dx=0.$$ \\
According to Prop 4.12, as we have 
$$g^2(x)\geq 0,$$
Therefore, $g(x)=0$ for all $x\in[a,b]$\\
\paragraph{(b)} Proof:\\
Firstly we justify that
$$\int_{a}^{b}(h(x)-\underline{h})dx=0.$$
Suppose $$H(x)=\int_{a}^{x}h(t)dt-\int_{a}^{x}\underline{h}dt,$$
then $$H(b)=\int_{a}^{b}(h(t)-\underline{h})dt=\int_{a}^{b}h(t)dt-\int_{a}^{b}\underline{h}dt.$$
Combining  with
$$\underline{h}=\frac{1}{b-a}\int_{a}^{b}h(t)dt,$$
we can find out that 
$$H(b)=\int_{a}^{b}h(t)dt-\int_{a}^{b}h(t)dt=\int_{a}^{b}(h(t)-\underline{h})dt=0.$$
Secondly we prove that $h$ is a constant function on $[a,b]$:
Suppose $v(x)=h(x)-\underline{h}$
\begin{align*}
    \int_{a}^{b}h(x)u(x)dx&=\int_{a}^{b}(v(x)+\underline{h})u(x)dx\\
    &=\int_{a}^{b}v(x)u(x)dx+\underline{h}\int_{a}^{b}u(x)dx\\
    &=\int_{a}^{b}v(x)u(x)dx
\end{align*}
According to the previous deduction, we have
$$\int_{a}^{b}v(x)dx=0$$
Then let $u(x)=v(x)$, then
$$\int_{a}^{b}v(x)u(x)=\int_{a}^{b}v^2(x)dx=0$$
According to Prop 4.12, $v^2(x)=0$ for all $x\in [a,b]$, namely 
$v(x)=0$. Recalling that $v(x)=h(x)-\underline{h}$, we have
$$h(x)=\underline{h}$$
Thus $h(x)$ is a constant function.

\section{}
Proof: \\
\paragraph{\romannumeral2 $\implies$\romannumeral1:}
\begin{align*}
    y(t)&=\cos t+\int_{0}^{t}s^2\sin(s-t)y(s)ds\\
    &=\cos t +\cos t\int_{0}^{t}s^2\sin s y(s)ds -\sin t\int_{0}^{t}
    s^2\cos sy(s)ds
\end{align*}
So by differentiating $y(t)$, we can get:
\begin{align*}
    y'(t)&=-\sin t +\left(-\sin t \int_{0}^{t}s^2\sin s\cdot y(s)ds+
    \sin t \cos t \cdot t^2 y(t)\right)-\left(\cos t \int_{0}^{t}
    s^2 \cos s\cdot y(s)ds+\sin t \cos t \cdot t^2 y(t)\right)\\
    &=-\sin t -\sin t \int_{0}^{t}s^2\sin s\cdot y(s)ds-
    \cos t \int_{0}^{t}s^2 \cos s \cdot y(s)ds
\end{align*}
Furthermore, we can differentiate $y'(t)$ and get second derivative of $y(t)$:
\begin{align*}
    y''(t)&=-\cos t-\left(\cos t\int_{0}^{t}s^2\sin s \cdot y(s)ds+
    \sin^2 t\cdot t^2y(t) \right)-\left(\sin t\int_{0}^{t}s^2\cos s\cdot y(s)ds+\cos^2 t \cdot t^2 y(t) \right)\\
    &=-\left(\cos t +\cos t\int_{0}^{t}s^2\sin s \cdot y(s)ds-\sin t\int_{0}^{t}s^2\cos s\cdot y(s)ds\right)-\left(
    (\sin^2 t+\cos^2 t)t^2y(t) \right)
\end{align*}
Recalling that 
$$ y(t)=\cos t +\cos t\int_{0}^{t}s^2\sin s y(s)ds -\sin t\int_{0}^{t}
s^2\cos sy(s)ds$$
$$\sin^2 t+\cos^2 t=1$$
We can rewrite $y''(t)$ as 
$$y''(t)=-y(t)-t^2y(t)$$
Also it's trivial to find that $y(0)=1,y'(0)=0$. Thus:
$$y''(t)+(1+t^2)y(t)=0 \;\; \forall t\in\mathbb{R} \text{, and that } y(0)=1,\, y'(0)=0$$

\paragraph{\romannumeral1$\implies$\romannumeral2:}
Denote
$$f(t)=\cos t\int_{0}^{t}s^2\sin s \cdot y(s)ds 
-\sin t\int_{0}^{t}s^2\cos s \cdot y(s)ds$$
Then it's similar to the calculation on the above to get:
$$f'(t)=-\sin t \int_{0}^{t}s^2\sin s\cdot y(s)ds-
\cos t \int_{0}^{t}s^2 \cos s \cdot y(s)ds$$
$$f''(t)=-\left(\cos t\int_{0}^{t}s^2\sin s \cdot y(s)ds-\sin t\int_{0}^{t}s^2\cos s\cdot y(s)ds\right)-
    t^2y(t)$$
Then rearrange the functions we have, we can easily find
$$\left\{
    \begin{aligned}
        f''(t)=-f(t)-t^2y(t)\\
        y''(t)=-y(t)-t^2y(t)
    \end{aligned}
\right.
$$
So we can find out
$$f''(t)-y''(t)=-\left(f(t)-y(t)\right)$$
Denote $\varphi (t)=f(t)-y(t)$, then $\varphi ''(x)=-\varphi (x)$.\\
Let $g(t)=\frac{\varphi (t)}{\cos t} \left(t\neq\frac{\pi}{2}+k\pi\right)$,
then differentiate $g(t)$, we can get:
$$g'(t)=\frac{\varphi '(x)\cos t+\sin t\cdot \varphi (t)}{\cos^2 t}$$
Denote $\lambda (x)=\varphi '(x)\cos t+\sin t \cdot \varphi (t)$, then
differentiate $\lambda(t)$, we can get:
\begin{align*}
    \lambda'(t)&=\varphi ''(t)\cos t-\varphi '(t)\sin t+\varphi '(t)\sin t 
    +\varphi(t)\cos t\\
    &=\cos t(\varphi(x)+\varphi ''(x))\\
    &=0
\end{align*}
Thus $\lambda(x)$ is a constant function.\\
Recalling that $f(0)=0,y(0)=1,f'(0)=0,y'(0)=0$, we have
$\varphi(0)=-1,\varphi'(0)=0$. \\
Thus $\lambda(0)=0$, which means
$g'(t)=0$.\\
$\therefore g(t)=\frac{\varphi (t)}{\cos t}$ is a constant function.\\
$\because \varphi(0)=-1,\cos 0=1$\\
$\therefore g(0)=-1,g(t)=\frac{\varphi (t)}{\cos t}=-1 \left(\left(t\neq\frac{\pi}{2}+k\pi\right)\right)$
$\therefore \varphi (t)=-\cos t \left(t\neq\frac{\pi}{2}+k\pi\right)$\\
For $t=\frac{\pi}{2}+k\pi(k\in\mathbb{Z})$, by the continuity of $\varphi (t)$, we can still get
$$\varphi (t)=-\cos t$$
Combining with $\varphi (t)=f(t)-y(t)$,
\begin{align*}
    y(t)&=f(t)-\varphi(t)\\
    &=\cos t +\cos t\int_{0}^{t}s^2\sin s y(s)ds -\sin t\int_{0}^{t}
    s^2\cos sy(s)ds\\
    &=\cos t+\int_{0}^{t}s^2\sin(s-t)y(s)ds \;\; \forall t\in\mathbb{R}
\end{align*}

\section{}
First, for such non-negative Riemann-integrable $f(x)$, we have:
$$\begin{aligned}
&\forall x \in [a,b] ,\, f(x) \ge 0 \\
\implies &\forall s,t \in [a,b] \text{ s.t. } s \le t , \\
&\int_s^t f(x)dx = \lim_{n\to\infty} L(P_n,f) = \lim_{n\to\infty} \sum_{k=1}^n \inf_{[x_{k-1} , x_k]} f \cdot (x_k - x_{k-1}) \ge \lim_{n\to\infty} \sum_{k=1}^n 0 \cdot (x_k - x_{k-1}) = 0
\end{aligned}$$
To prove the negation wrong. Suppose $\exists x_0$ on which $f(x_0)$ is continuous, we have $f(x_0)>0$:
$$\begin{aligned}
&\lim_{x\to x_0} f(x) = f(x_0) \\
\implies &\exists \delta_0>0 \text{ s.t. } \forall x\in(x_0 - \delta_0 ,\, x_0 + \delta_0) ,\, \left| f(x) - f(x_0) \right| < \frac{1}{2} f(x_0) 
\end{aligned}$$
$$\begin{aligned}
\implies \forall x\in(x_0 - \delta_0 ,\, &x_0 + \delta_0) ,\, f(x) > \frac{1}{2} f(x_0) \\
\implies \int_{x_0 - \delta_0}^{x_0 + \delta_0} f(x) dx &= \lim_{n\to\infty} L(P_n,f) = \lim_{n\to\infty} \sum_{k=1}^n \inf_{[x_{k-1} ,\, x_k]} f \cdot (x_k - x_{k-1}) \\
&\ge \lim_{n\to\infty} \frac{1}{2} f(x_0) \sum_{k=1}^n (x_k - x_{k-1}) = \delta_0 \cdot f(x_0) > 0
\end{aligned}$$
In such case:
$$\begin{aligned}
\int_a^b f(x) dx &= \int_a^{x_0 - \delta_0} f(x) dx + \int_{x_0 - \delta_0}^{x_0 + \delta_0} f(x) dx + \int_{x_0 + \delta_0}^b f(x) dx \\
&\ge 0 + \int_{x_0 - \delta_0}^{x_0 + \delta_0} f(x) dx + 0 \\
&>0
\end{aligned}$$
which contradicts with $\int_a^b f(x)dx = 0$. Therefore, our supposition is wrong, finishing the prove that $\forall x_0$ on which $f(x_0)$ is continuous, we have $f(x_0)=0$.

\section{}
\paragraph{(a)}
\def\cost{\cos\theta}
\def\costt{\cos\frac{\theta}{2}}
\def\tantt{\tan\frac{\theta}{2}}
\def\sectt{\sec\frac{\theta}{2}}
\def\sintt{\sin\frac{\theta}{2}}
\begin{flalign*} 
	\;&\frac{1-a\cost}{1-2a\cost+a^2}\\
	=&\frac{\frac{1}{2}-a\cost+\frac{a^2}{2}+\frac{1}{2}-\frac{a^2}{2}}{1-2a\cost+a^2}\\
	=&\frac{1}{2}+\frac{\frac{1}{2}-\frac{a^2}{2}}{1-2a\cost+a^2}\\
	=&\frac{1}{2}+\frac{1}{2}\cdot (1-a)(1+a)\cdot \frac{1}{(\costt)^2+(\sintt)^2-2a\cdot((\costt)^2-(\sintt)^2)+a^2\cdot((\sintt)^2+(\costt)^2)}\\
	=&\frac{1}{2}+\frac{1}{2}\cdot (1-a)(1+a)\cdot \frac{1}{(\sintt)^2\cdot (a+1)^2+(\costt)^2\cdot(1-a)^2}\\
	=&\frac{1}{2}+\frac{1}{2}\cdot(1-a)(1+a)\cdot \frac{(\sectt)^2}{(a-1)^2+(\tantt)^2\cdot(a+1)^2}\\
	=&\frac{1}{2}+\frac{1-a}{1+a}\cdot \frac{\frac{1}{2}(\sectt)^2}{(\frac{1-a}{1+a})^2+(\tantt)^2}
\end{flalign*}

\paragraph{(b)}
From (a) we know:
\def\tempa{\frac{1-a\cost}{1-2a\cost+a^2}}
\def\tempb{\frac{1-a}{1+a}\cdot \frac{\frac{1}{2}(\sectt)^2}{(\frac{1-a}{1+a})^2+(\tantt)^2}}
\def\tempc{\frac{\frac{1}{2}(\sectt)^2}{(\frac{1-a}{1+a})^2+(\tantt)^2}}
\def\a{\alpha}
\def\b{\beta}
\def\intab{\int_{\alpha}^{\beta}}
$$\tempa = \frac{1}{2}+\tempb$$
Hence
\begin{flalign*} 
\intab \tempa \d x&=\intab \left(\frac{1}{2}+\tempb\right)\d x\\
\implies \intab \tempa \d x&=\frac{\b-\a}{2}+\frac{1-a}{1+a}\intab \tempc\d x 
\end{flalign*}
Next we evaluate $\intab \tempc \d x$:
\\Let $u=\tantt$
\begin{flalign*} 
	\;&\intab \tempc \d x\\
	=&\intab \frac{1}{(\frac{1-a}{1+a})^2+(\tantt)^2}\d \left(\tantt\right)\\
	=&\int_{\tan\frac{\a}{2}}^{\tan\frac{b}{2}}\frac{1}{(\frac{1-a}{1+a})^2+u} \d u\\
	=&\left[ \frac{1+a}{1-a}\tan^{-1}\left(\frac{1+a}{1-a}u\right)\right]_{\tan\frac{\a}{2}}^{\tan\frac{\b}{2}}
\end{flalign*}
hence
$$\tempa=\frac{\b-\a}{2}+\tan^{-1}\left(\frac{1+a}{1-a}\tan \frac{\b}{2}\right)-\tan^{-1}\left(\frac{1+a}{1-a}\tan \frac{a}{2}\right)$$
\paragraph{(c)}~{}
\def\cons{\frac{1-a}{1+a}}
\def\fcons{\frac{1+a}{1-a}}
\def\e{\epsilon}
\def\d{\delta}
\def\dx{\mathrm{d}x}
\\\\First we evaluate the case when $0<a<1$. In this case, $0<\cons<1$ and $\fcons>1$.
\\Choose $\e>0$.
\\From 1023 we know that $\lim\limits_{x\to \frac{\pi}{2}^-}\tan\theta=+\infty $, hence $\lim\limits_{\b \to \pi^-}\fcons \tan \frac{\b}{2}=+\infty$
\\hence, denote $\tan\left(\frac{\pi}{2}-\frac{\e}{3}\right)=K$, then
$$\exists \d_1>0 \st \tif -\d_1<\b-\pi<0,~ \then ~\fcons\tan\frac{\b}{2}>K $$
\\Moreover, we know that $f(x):=\tan^{-1}(x)$ monotonely increases on $\R$, hence
$$\tan^{-1}\left(\fcons \tan\frac{\b}{2}\right)>\tan^{-1}(K)\implies \tan^{-1}\left(\fcons \tan\frac{\b}{2}\right)>\frac{\pi}{2}-\e$$
\\Similarly, $\lim\limits_{\theta \to 0}\tan\theta=0$ and $\lim\limits_{x\to 0}\tan^{-1}x=0$. Hence by composition rules
$$\lim\limits_{\a\to 0}\tan^{-1}\left(\fcons \tan\frac{\a}{2}\right)=\lim\limits_{y\to 0}\tan^{-1}y=0$$
hence 
$$\exists \d_2 \st \tif 0<\a-0<\d_2,~\then ~\left|\tan^{-1}\left(\fcons\tan\frac{\a}{2}\right)-0\right|<\frac{\e}{3}\implies \tan^{-1}\left(\fcons\tan\frac{\a}{2}\right)<\frac{\e}{3}$$
Choose $\a_1, \b_1$ that $0<\a_1<\d_2$ and $\pi-\d_1<\b_1<\pi$,then
$$\tan^{-1}\left(\frac{1+a}{1-a}\tan \frac{\b_1}{2}\right)-\tan^{-1}\left(\frac{1+a}{1-a}\tan \frac{a_1}{2}\right)>\frac{\pi}{2}-\frac{\e}{3}-\frac{\e}{3}>\frac{\pi}{2}-\e$$
This shows
$$\cons\int_{\a_1}^{\b_1}\tempc \dx=\cons\left[ \frac{1+a}{1-a}\tan^{-1}\left(\frac{1+a}{1-a}u\right)\right]_{\tan\frac{\a_1}{2}}^{\tan\frac{\b_1}{2}}>\frac{\pi}{2}-\e$$
Moreover, from \textbf{(a)} we know 
\begin{flalign*}
\tempa &= \frac{1}{2}+\tempb\\\implies \tempa-\frac{1}{2}&=\tempb
\end{flalign*}
hence
$$\int_{\a_1}^{\b_1}\left(\tempa-\frac{1}{2}\right)\dx>\frac{\pi}{2}-\e$$
Because $0<a<1$, $\frac{1-a\cost}{1-2a\cost+a^2}=\frac{1-a\cost}{(a-1)^2+2a(1-\cost)}>0$, hence
$$\int_{0}^{\pi}\left(\tempa-\frac{1}{2}\right)\dx>\int_{\a_1}^{\b_1}\left(\tempa-\frac{1}{2}\right)\dx>\frac{\pi}{2}-\e$$
hence
$$\int_{0}^{\pi}\tempa\dx>\frac{\pi}{2}+\frac{\pi}{2}-\e=\pi-\e$$
Above arguments hold for any $\e>0$, thus
\begin{align} 
\forall \e>0 , \int_{0}^{\pi}\tempa\dx>\frac{\pi}{2}\implies \int_{0}^{\pi}\tempa \dx\ge\pi
\end{align}
\\Next we'll show that $$\int_{0}^{\pi}\tempa\dx\le \pi$$
Consider $\tan^{-1}\left(\fcons\tan\frac{\b}{2}\right)$ and $\tan^{-1}\left(\fcons\tan\frac{\a}{2}\right)$ again:
\\According to the definition of $\tan^{-1}$, we know that $\tan^{-1}\left(\fcons\tan\frac{\b}{2}\right)<\frac{\pi}{2}$
\\Moreover, given $\a>0$, $\cons\tan\frac{\a}{2}>0$, and hence $\tan^{-1}\left(\fcons\tan\frac{\a}{2}\right)>0$ 
\\Thus, 
$$\cons\left[ \frac{1+a}{1-a}\tan^{-1}\left(\frac{1+a}{1-a}u\right)\right]_{\tan\frac{\a}{2}}^{\tan\frac{\b}{2}}<\frac{\pi}{2}-0=\frac{\pi}{2}$$
\\Then consider $\tempa $:
$$\tempa\le \frac{1-a\cost}{(a\cost)^2-2a\cost+1}=\frac{1-a\cost}{(1-a\cost)^2}=\frac{1}{1-a\cost}<\frac{1}{1-a}$$
hence $\tempa $ is bounded above by $M:=\frac{1}{1-a}$
\\\\hence, for any $\e>0$,
\begin{flalign*}
\;&\int_{0}^{\pi}\tempa \dx\\
=&\int_{0}^{\frac{\e}{3M}}\tempa\dx+\int_{\frac{\e}{3M}}^{\pi-\frac{\e}{3M}} \tempa \dx+\int_{\pi-\frac{\e}{3M}}^{\pi}\tempa\dx\\
<&\frac{\e}{3}+\pi\cdot\frac{1}{2}+\cons\left[\frac{1}{2}+ \frac{1+a}{1-a}\tan^{-1}\left(\frac{1+a}{1-a}u\right)\right]_{\tan\frac{\a_1}{2}}^{\tan\frac{\b_1}{2}}+\frac{\e}{3}\\
<&\frac{2\e}{3}+\frac{\pi}{2}+\frac{\pi}{2}\\
<&\e+\pi
\end{flalign*}
hence
\begin{align}
	\int_{0}^{\pi} \tempa \dx\le \pi
\end{align}
combining inequalities (1) and (2):
$$\int_{0}^{\pi}\tempa \dx=\pi$$
Next we discuss the case where $a>1$.
\def\tempb{\frac{1-b\cost}{1-2b\cost+b^2}}
\\\\When $a>1$, denote $b=\frac{1}{a}$, then:
$$\tempa+\tempb=\tempa+\frac{a^2-a\cost}{a^2-2a\cost+1}=1$$
moreover, since $b<1$, we have 
$$\int_{0}^{\pi}\tempb \dx=\pi$$
hence
$$\int_{0}^{\pi}\tempa\dx=\int_{0}^{\pi}\left(1-\tempb\right)\dx=\pi-\pi=0$$
In conclusion,
$$I_{\a}=\int_{0}^{\pi}\tempa\dx=\left\{
\begin{aligned}
	\pi &\quad\tif 0<a<1\\
	0   &\quad\tif a>1
\end{aligned}
\right.
$$

\end{document}