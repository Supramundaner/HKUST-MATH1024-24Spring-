\documentclass[11pt]{article}
\usepackage{geometry, amsmath, amsthm, amssymb, graphicx, amsfonts, yhmath, setspace, float, subfigure}
\usepackage[table]{xcolor}
\usepackage{fullpage}
\usepackage{enumitem}
\usepackage{tikz-cd}
\usepackage{setspace}
\setlength{\parindent}{0.0in}
\setlength{\parskip}{12pt}
\newtheorem{Thm}{Theorem}[section]
\newtheorem{Lem}[Thm]{Lemma}
\newtheorem{Prop}[Thm]{Proposition}
\newtheorem{Cor}[Thm]{Corollary}
\newtheorem{Rem}[Thm]{Remark}
\newtheorem{Def}[Thm]{Definition}
\def\sep{\vspace{1cm}\hrule\vspace{1cm}}
\def\tab{\;\;\;\;\;\;}
\def\R{\mathbb{R}}
\def\Q{\mathbb{Q}}
\def\N{\mathbb{N}}
\def\C{\mathbb{C}}
\def\Z{\mathbb{Z}}
\def\a{\alpha}
\def\b{\beta}
\def\c{\gamma}
\def\d{\delta}
\def\e{\epsilon}
\def\h{\theta}
\def\w{\omega}
\def\W{\Omega}
\def\bu{\mathbf{u}}
\def\bv{\mathbf{v}}
\def\iff{\Longleftrightarrow}
\def\to{\rightarrow}
\def\inj{\hookrightarrow}
\def\surj{\twoheadrightarrow}
\def\imply{\Longrightarrow}
\def\x{\times}
\def\<{\langle}
\def\>{\rangle}
\def\oo{\infty}
\def\normal{\triangleleft}
\def\h{\hspace*{0.5cm}}
\title{\LaTeX{} Answer to Problem Set \#}
\author{AO Yuzhuo, JIN Zhibo, LYU Changlai\thanks{names listed in alphabetical order}}
\date{\today}

\begin{document}
\maketitle

\section{}
\paragraph{(a)}
Using such definition: $n,j \in \mathbb{N} \Rightarrow \dfrac{j}{n} \in \mathbb{Q} ,\, \dfrac{\sqrt{2}j}{n} \notin \mathbb{Q} ,\, (\sqrt{2}-1) \cdot \dfrac{j}{n}+1 \notin \mathbb{Q}$

$$\begin{aligned}
\text{LHS} &= \lim\limits_{n\to\infty} \sum\limits_{j=1}^n \dfrac{\sqrt{2}}{n} \cdot \chi_\mathbb{Q} \left( \dfrac{\sqrt{2}}{n} \cdot j \right) \\
&= \lim\limits_{n\to\infty} \sum\limits_{j=1}^n \dfrac{\sqrt{2}}{n} \cdot 0 \\
&= 0
\\\\
\text{RHS} &= \lim\limits_{n\to\infty} \sum\limits_{j=1}^n \dfrac{1}{n} \cdot \chi_\mathbb{Q} \left( \dfrac{1}{n} \cdot j \right) + \lim\limits_{n\to\infty} \sum\limits_{j=1}^n \dfrac{\sqrt{2}-1}{n} \cdot \chi_\mathbb{Q} \left( \dfrac{\sqrt{2}-1}{n} \cdot j \right) \\
&= \lim\limits_{n\to\infty} \sum\limits_{j=1}^n \dfrac{1}{n} \cdot 1 + \lim\limits_{n\to\infty} \sum\limits_{j=1}^n \dfrac{\sqrt{2}-1}{n} \cdot 0 \\
&= 1
\end{aligned}$$
LHS $\ne$ RHS, conflict.

\section{}
\paragraph{Case 1}

Red, Left, Inner Measure:
$$\begin{aligned}
S_n &:= \bigcup_{j=1}^n \left< \dfrac{c}{n} \cdot j ,\, a - \dfrac{a-b-c}{n} \cdot j \right> \times \left< \dfrac{h}{n} \cdot (j-1) ,\, \dfrac{h}{n} \cdot j \right> \\
\lim\limits_{n\to\infty} S_n &= \lim\limits_{n\to\infty} \sum_{j=1}^n \left( a - \dfrac{a-b}{n} \cdot j \right) \\
&= \lim\limits_{n\to\infty} \left( ah - \dfrac{(a-b)h}{n^2} \sum_{j=1}^n j \right) \\
&= \lim\limits_{n\to\infty} \left( ah - \dfrac{(a-b)h}{n^2} \cdot \dfrac{n^2+n}{2} \right) \\
&= \dfrac{a+b}{2} \cdot h
\end{aligned}$$

\begin{figure}[H]
\centering
\includegraphics[width=0.45\linewidth]{1.pdf}
\hfill
\includegraphics[width=0.45\linewidth]{2.pdf}
\end{figure}

Blue, Right, Outer Measure:
$$\begin{aligned}
T_n &:= \bigcup_{j=1}^n \left< \dfrac{c}{n} \cdot (j-1) ,\, a - \dfrac{a-b-c}{n} \cdot (j-1) \right> \times \left< \dfrac{h}{n} \cdot (j-1) ,\, \dfrac{h}{n} \cdot j \right> \\
\lim\limits_{n\to\infty} T_n &= \lim\limits_{n\to\infty} \sum_{j=1}^n \left( a - \dfrac{a-b}{n} \cdot (j-1) \right) \\
&= \lim\limits_{n\to\infty} \left( ah - \dfrac{(a-b)h}{n^2} \cdot \dfrac{n^2-n}{2} \right) \\
&= \dfrac{a+b}{2} \cdot h
\end{aligned}$$

We get $\lim\limits_{n\to\infty} S_n = \lim\limits_{n\to\infty} T_n$.

\paragraph{Case 2}

Red, Left, Inner Measure:
$$\begin{aligned}
S_n &:= \bigcup_{j=1}^n \left< \dfrac{c}{n} \cdot j ,\, a + \dfrac{b+c-a}{n} \cdot (j-1) \right> \times \left< \dfrac{h}{n} \cdot (j-1) ,\, \dfrac{h}{n} \cdot j \right> \\
\lim\limits_{n\to\infty} S_n &= \lim\limits_{n\to\infty} \sum_{j=1}^n \left( a + \dfrac{a-b}{n} \cdot (j-1) - \dfrac{c}{n} \right) \\
&= \lim\limits_{n\to\infty} \left( ah + \dfrac{(b-a)h}{n^2} \cdot \dfrac{n^2-n}{2} -\dfrac{ch}{n} \right) \\
&= \dfrac{a+b}{2} \cdot h
\end{aligned}$$

\begin{figure}[H]
\centering
\includegraphics[width=0.48\linewidth]{3.pdf}
\hfill
\includegraphics[width=0.48\linewidth]{4.pdf}
\end{figure}

Blue, Right, Outer Measure:
$$\begin{aligned}
T_n &:= \bigcup_{j=1}^n \left< \dfrac{c}{n} \cdot (j-1) ,\, a + \dfrac{b+c-a}{n} \cdot j \right> \times \left< \dfrac{h}{n} \cdot (j-1) ,\, \dfrac{h}{n} \cdot j \right> \\
\lim\limits_{n\to\infty} T_n &= \lim\limits_{n\to\infty} \sum_{j=1}^n \left( a + \dfrac{b-a}{n} \cdot j +\dfrac{c}{n} \right) \\
&= \lim\limits_{n\to\infty} \left( ah + \dfrac{(b-a)h}{n^2} \cdot \dfrac{n^2+n}{2} + \dfrac{ch}{n} \right) \\
&= \dfrac{a+b}{2} \cdot h
\end{aligned}$$

We also get $\lim\limits_{n\to\infty} S_n = \lim\limits_{n\to\infty} T_n$.

\paragraph{In either case:}
$$\begin{aligned}
&\because \forall n \in \mathbb{N}_+ ,\, S_n \le \mu_*(\Omega) \le \mu^*(\Omega) \le T_n \\
&\therefore \lim\limits_{n\to\infty} S_n \le \mu_*(\Omega) \le \mu^*(\Omega) \le \lim\limits_{n\to\infty} T_n \\
&\text{With the fact that: } \lim\limits_{n\to\infty} S_n = \lim\limits_{n\to\infty} T_n = \dfrac{a+b}{2} \cdot h \\
&\therefore \mu_*(\Omega) = \mu^*(\Omega) = \dfrac{a+b}{2} \cdot h \\
&\therefore \mu(\Omega) = \dfrac{a+b}{2} \cdot h
\end{aligned}$$

\section{}
\paragraph{(a)Proof:}\ \\Firstly, we show that $\Omega_1 \cap \Omega_2$ is Jordan
measurable:\\We use Proposition 4.4. Given that $\Omega_1$ and $\Omega_2$ are Jordan measurable, there exist sequences
of inner simple regions $\{S_n^{(i)}\}^{\infty }_{n=1}$
and outer simple regions $\{T_n^{(i)}\}^\infty_{n=1}$, where i = 1, 2, such that
$\{S_n^{(i)}\}^{\infty }_{n=1}\subset \Omega_i \subset \{T_n^{(i)}\}^\infty_{n=1}$  for each $n \in N$ and  $i \in {1, 2}$, and
$$\lim_{n\to\infty}A(S_n^{(i)})=\lim_{n\to\infty}A(T_n^{(i)})=\mu(\Omega_i)$$
By induction, for two simple regions $X\subset Y$,
it's intuitive that $X\cap Y$, $X\cup Y$ and $X\slash Y$ are all simple regions.
From elementary set theory, we know 
$$(T_n^{(1)}\cap T_n^{(2)})\slash (S_n^{(1)}\cap S_n^{(2)}) \subseteq 
\left((T_n^{(1)}\cap T^{(2)}_2)\slash S_n^{(1)}\right)\cup\left(
    (T_n^{(1)}\cap T_n^{(2)})\slash S_n^{(2)}
\right)
\subseteq (T_n^{(1)}\slash S_n^{(1)})\cup (T_n^{(2)}\slash S_n^{(2)})$$
Therefore, we get:
$$A\left((T_n^{(1)}\cap T_n^{(2)})\slash (S_n^{(1)}\cap S_n^{(2)})\right)\leq
A\left((T_n^{(1)}\slash S_n^{(1)})\cup (T_n^{(2)}\slash S_n^{(2)})\right)\leq 
A(T_n^{(1)}\slash S_n^{(1)})+A(T_n^{(2)}\slash S_n^{(2)})
$$
which means that:
$$0\leq A(T_n^{(1)}\cap T_n^{(2)})-A (S_n^{(1)}\cap S_n^{(2)})\leq
 A(T_n^{(1)})+A(T_n^{(2)})-A(S_n^{(1)})-A(S_n^{(2)})$$
 For we have $$\lim_{n\to\infty}\left(A(T_n^{(1)})+A(T_n^{(2)})-
 A(S_n^{(1)})-A(S_n^{(2)})\right)=0$$,
 by squeeze theorem and Bolzano-Weierstrass Theorem, one can take a
 convergent subsequence $ \{A(T_{n_j}^{(1)}\cap T_{n_j}^{(2)})\}$ of
$\{A(T_{n}^{(1)}\cap T_{n}^{(2)})\}$ , and then 
$ \{A(S_{n_j}^{(1)}\cap S_{n_j}^{(2)})\}$ also  has a convergent
sub-subsequence
$ \{A(S_{n_{j_k}}^{(1)}\cap S_{n_{j_k}}^{(2)})\}$, then both two sub-subsequences' limits exist.
 
We know that $T_{n_{j_k}}^{(1)}\cap T_{n_{j_k}}^{(2)} \supset \Omega_1\cap\Omega_2$ and 
$S_{n_{j_k}}^{(1)}\cap S_{n_{j_k}}^{(2)} \subset \Omega_1\cap\Omega_2$.\\
By Proposition 4.4, $\Omega_1\cap\Omega_2$ is Jordan measurable with
$$\mu (\Omega_1\cap \Omega_2)=\lim_{k\to\infty}A (S_{n_k}^{(1)}\cap S_{n_k}^{(2)})$$
\\ \\
Secondly, we will show that $\Omega_1\slash \Omega_2$ is Jordan measurable:\\
We have known that:
$$\Omega_1\slash \Omega_2\equiv \Omega_1\slash (\Omega_1\cap\Omega_2),(\Omega_1\cap\Omega_2)\subset\Omega_1$$
Let $\Omega_3\equiv (\Omega_1\cap\Omega_2)$.
Given that $\Omega_1$ and $\Omega_3$ are Jordan measurable, there exist sequences
of inner simple regions $\{S_n^{(i)}\}^{\infty }_{n=1}$
and outer simple regions $\{T_n^{(i)}\}^\infty_{n=1}$, where $i = 1, 3$ (which is different from the above $T_n$ and $S_n$), such that
$\{S_n^{(i)}\}^{\infty }_{n=1}\subset \Omega_i \subset \{T_n^{(i)}\}^\infty_{n=1}$  
for each $n \in N$ and  $i \in \{1, 3\}$, and

$$\lim_{n\to\infty}A(S_n^{(i)})=\lim_{n\to\infty}
A(T_n^{(i)})=\mu(\Omega_i)$$

Then by $T_n^{(1)}\supset S_n^{(3)}$,  we can easily find that
$$ A(T_n^{(1)})-A(S_n^{(3)})= A(T_n^{(1)}\slash S_n^{(3)})\geq A(S_n^{(1)}\slash T_n^{(3)}) \geq A(S_n^{(1)})-A(T_n^{(3)}) $$
As we know, $\Omega_1$ \& $\Omega_3$ are both Jordan measurable:
$$\lim_{n\to\infty}A(T_n^{(1)})-A(S_n^{(3)})=\lim_{n\to\infty}A(S_n^{(1)})-A(T_n^{(3)})=\mu(\Omega_1)-\mu(\Omega_3)$$
then by squeeze theorem we have
$$\lim_{n\to\infty} A(T_n^{(1)}\slash S_n^{(3)})=\lim_{n\to\infty} A(S_n^{(1)}\slash T_n^{(3)})$$


$\because T_n^{(1)}\slash S_n^{(3)}\supset \Omega_1\slash\Omega_3$ and 
$S_n^{(1)}\slash T_n^{(3)}\subset \Omega_1\slash\Omega_3$\\
$\therefore $ By Proposition 4.4, $\Omega_1\slash\Omega_3$ is Jordan measurable\\
Recall that $\Omega_3\equiv (\Omega_1\cap\Omega_2)$, 
we finally get $\Omega_1\slash\Omega_2$ is Jordan measurable. (Similarly, $\Omega_2\slash\Omega_1$ is Jordan measurable)



\paragraph{(b)Proof:}\ \\We have known that 
$$\Omega_1\cup \Omega_2\equiv (\Omega_1\slash \Omega_2)
\cup(\Omega_1\cap\Omega_2)\cup(\Omega_2\slash\Omega_1) $$
and by finite additivity we get:
$$\mu(\Omega_1)=\mu(\Omega_1\slash \Omega_2)+\mu(\Omega_1\cap\Omega_2)$$
$$\mu(\Omega_2)=\mu(\Omega_2\slash \Omega_1)+\mu(\Omega_1\cap\Omega_2)$$
$$\mu(\Omega_1\cup \Omega_2)=\mu(\Omega_1\slash \Omega_2)
+\mu(\Omega_1\cap\Omega_2)+\mu(\Omega_2\slash\Omega_1) $$
So finally we have:
$$\mu(\Omega_1\cup\Omega_2)=\mu(\Omega_1)+\mu(\Omega_2)-\mu(\Omega_1\cap\Omega_2)$$



\paragraph{(c)Proof:}\ \\Recalling the second part of 3(a), we have got:
$$ \lim_{n\to\infty}A(T_n^{(1)}\slash S_n^{(3)})
=\lim_{n\to\infty}A(S_n^{(1)}\slash T_n^{(3)})=\mu(\Omega_1\slash\Omega_3)$$
when $\Omega_3\subset\Omega_1$,which means that
$$\mu(\Omega_1\slash\Omega_3)=\mu(\Omega_1)-\mu(\Omega_3)$$
It's equivlent to show that if $\Omega_1\subset \Omega_2$, then 
$\mu(\Omega_2\slash\Omega_1)=\mu(\Omega_2)-\mu(\Omega_1)$

\section{}
Let $\W$ be a circle with radius r.
\\Consider two sequences of regular polygons $\{E_n\}$ and $\{F_n\}$, where $E_n$ denotes an n-side regular polygon with all its endpoints on circle $\W$, and $F_n$ denotes an n-side regular polygon with all its sides tangent to the circle (i.e. the radius of the circle is the height of the triangle formed by connecting the center with two neighboring endpoints).
\\Hence, by constructions, we have $E_n \subseteq \W \subseteq F_n$
\\Moreover, we've shown that triangles are Jordan measurable, and the translational invariance holds.
\\Hence by finite additivity and the fact that polygons can be reduced to finitely many triangles, we have:
$$\forall n\in \N \text{ and } n\ge 3, E_n, F_n \text{ are both Jordan measurable.}$$
\def\muE{\mu(E_n)}
\def\muF{\mu(F_n)}
To find $\muE$, we divide $E_n$ into $n$ triangles by connecting the center of this circle with all the end points of $E_n$.
\\Denote this triangles by $H_1, H_2, \cdots ,H_n $
\\The Jordan measure for triangles is base times height divide by 2 (proved before), hence:
$$\forall i \in \{i \big| i\in \N\text{ and }1\le i\le n\}, \mu(H_i) = \frac{1}{2} \cdot r\cdot r\cdot \sin \frac{2\pi}{n} = \frac{r^2\sin \left(\frac{2\pi}{n}\right)}{2}$$
By finite additivity again,
$$\muE=n\cdot \frac{r^2\sin\left(\frac{2\pi}{n}\right)}{2}\cdot$$
Let $y=\frac{2\pi}{n}$. By composition rules and the fact that $\lim\limits_{x\to 0}\frac{\sin x}{x}=1$:
$$\lim\limits_{n\to \infty}\muE=\lim\limits_{n\to \infty} \left( \frac{1}{\frac{2\pi}{n}}\cdot \sin \left( \frac{2\pi}{n} \right)\cdot \left(\frac{r^2}{2}\right)\cdot 2\pi \right)=\pi r^2\lim\limits_{y\to 0}\frac{\sin y}{y}=\pi r^2$$
To find $\muF$, we use the same method to cut $F_n$ into $n$ triangles $I_1, I_2,......,I_n$, then:
$$\forall i\in \{i \big| i\in \N \text{ and } 1\le i\le 3\}, \mu(H_i)=r\cdot r\cdot \tan\left(\frac{2\pi}{2n}\right)\cdot2\cdot \frac{1}{2}=r^2\tan\left(\frac{\pi}{n}\right)$$
Thus, $\muF=n\cdot r^2\tan (\frac{\pi}{n})$
\\Thus
$$\lim\limits_{n\to \infty} \muF= \lim\limits_{n\to \infty} \left( nr^2\tan \left(\frac{\pi}{n} \right)\right)=\lim\limits_{n\to\infty} \left(\pi r^2 \cdot \dfrac{\sin \left(\frac{\pi}{n}\right)}{\frac{\pi}{n}\cos\left(\frac{\pi}{n}\right)}\right)=\pi r^2\cdot \frac{1}{1}=\pi r^2$$
Thus by the result of \textbf{Exercise 4.8} we conclude that:
$$\mu(\W)=\pi r^2$$

\section{}
\paragraph{Proof:} \ \\
Let $f(x)$ is a monotone bounded function on $[a, b]$,
 by Equation 4.1 and Isometric Invariance of Jordan measure, without loss of generality, we can 
 assume that $f(a)=0$ and $f(x)$ is monotone increasing on $[a,b]$, so
 $\forall x\in [a,b],f(x)\in [0,f(b)].$\\
 We will construct a sequence of patitions $P_n$ such that
 $U(P_n,f)$ and $L(P_n,f)$ converge to the same limit.\\
 Let $P_n$ be the partition $$
 x_0:=a<a+\frac{b-a}{n}<a+\frac{2(b-a)}{n}<.
 ..<a+\frac{(n-1)(b-a)}{n}<a+\frac{n(b-a)}{n}=b=:x_n$$
Then, for any $i\in \{1,...,n\}$ we have
$$\begin{aligned}
M_i&:= \sup_{x\in [x_{i-1},x_i]} f(x)=f(x_i) \\
m_i&:= \inf_{x\in [x_{i-1},x_i]} f(x)=f(x_{i-1})
\end{aligned}$$
Hence,
$$
U(P_n,f)=\sum_{i=1}^{n}M_i\cdot \frac{b-a}{n}
=\sum_{i=1}^{n}f(x_i)\cdot\frac{b-a}{n}$$
$$
L(P_n,f)=\sum_{i=1}^{n}m_i\cdot \frac{b-a}{n}
=\sum_{i=1}^{n}f(x_{i-1})\cdot\frac{b-a}{n}$$
Firstly, we can easily find that:
$$U(P_n,f)-L(P_n,f)=(f(x_n)-f(x_0))\cdot\frac{b-a}{n}=\frac{f(b)(b-a)}{n}$$
Let $n\to\infty$,then:
$$\lim_{n\to\infty}U(P_n,f)-L(P_n,f)=0$$
Next, we will show that a subsequence of $U(P_n,f)$ converges:

\begin{align*}    
U(P_{2^{n+1}},f)&=\sum_{i=1}^{2^{n+1}}M_i\cdot \frac{b-a}{2^{n+1}}
\\&=\sum_{i=1}^{2^{n+1}}f(x_i)\cdot\frac{b-a}{2^{n+1}}\\
&= \sum_{i=1}^{2^{n}}f(x_{2i})\cdot\frac{b-a}{2^{n+1}}+
\sum_{i=1}^{2^{n}}f(x_{2i-1})\cdot\frac{b-a}{2^{n+1}}
\end{align*}
In this case 
\begin{align*}    
    U(P_{2^{n}},f)=\sum_{i=1}^{2^n}f(x_{2i})\cdot\frac{b-1}{2^n}
\end{align*}
Thus 
$$U(P_{2^{n+1}},f)-U(P_{2^n},f)=\sum_{i=1}^{2^{n}}(f(x_{2i-1})-f(x_{2i}))\cdot\frac{b-a}{2^{n+1}}<0$$
Therefore $\{U(P_{2^n},f)\}$ is a bounded and monotone decreasing sequence, the limit of 
$\{U(P_{2^n},f)\}$(which is a subsequence of $\{U(P_n,f)\}$) exists.
As 
$\{U(P_{2^n},f)-L(P_{2^n},f)\}$ is a subsequence of $\{U(P_n,f)-L(P_n,f)\}$, 
$$\lim_{n\to\infty} U(P_{2^n},f) - L(P_{2^n},f) = 0$$
$$\imply \lim_{n\to\infty} L(P_{2^n},f) = \lim_{n\to\infty} U(P_{2^n},f) - \left( U(P_{2^n},f) - L(P_{2^n},f) \right) = \lim_{n\to\infty} U(P_{2^n},f)$$
According to Proposition 4.7, $f$ is Riemann integrable.

\section{}
Here we follow the logical order (\romannumeral1) $\to$ (\romannumeral2) $\to$ (\romannumeral3) $\to$ (\romannumeral4) $\to$ (\romannumeral5) $\to$ (\romannumeral1).
\paragraph{(i) $\imply$ (ii): }~{}
\\\\
Let $\varepsilon=\frac{1}{n}>0$, where $n$ can be any positive integer. 
\\Thus by given condition, there exists a partition $P_n$ such that :
$$U(P_n, f)-L(P_n, f)<\varepsilon=\frac{1}{n}$$
Then we consider the sequence $\{U(P_n, f)-L(P_n, f)\}$
\\Because: 
$$0\le U(P_n, f)-L(P_n, f) < \frac{1}{n}$$
Thus, by squeeze theorem (and the fact that $\lim\limits_{n\to \oo}\frac{1}{n}=0$),
\\ we've found a sequence of partitions $\{P_n\}$ that: 
$$\lim\limits_{n\to \oo}\big(U(P_n, f)-L(P_n, f)\big)=0$$
\paragraph{(ii) $\imply$ (iii):}~{}
\\\\
Consider the sequence of partition ${P_n}$ mentioned in (ii): we'll show that ${U(P_n,f)}$ is bounded.
\\ By definition, for any positive integers $n$ and $m$ , we have $U(P_n, f)\ge L(P_m,f)$
\\ Let $m=1$, then: $U(P_n, f)\ge L(P_1, f)$, which shows ${U(P_n,f)}$ is bounded below by $L(P_1,f)$.
\\ Next, given $\left(U(P_n,f)-L(P_n,f)\right)$ converges to 0, choose $\varepsilon=1$, then:
$$\exists N\in \N \text{ and } N>0, \,s.t.\,  \forall n>N,   U(P_n, f)-L(P_n, f)<1$$
Let $K=\max\{1, \big(U(P_1, f)-L(P_1, f)\big),......, \big(U(P_N, f)-L(P_N, f)\big)\}$, thus for any $n\in \N$:
$$U(P_n, f)-L(P_n, f)\le K\imply U(P_n, f)\le L(P_n, f)+K\le \sup\limits_{x\in[a,b]}f(x)\cdot (b-a)+K $$
This shows $\{U(P_n,f)\}$ is also bounded above.
\\Hence, $\{U(P_n, f)\}$ is bounded, and $\textit{BWT}$ can be applied: we can found a convergent subsequence of $\{U(P_n, f)\}$.
\\Denote this subsequence by $\{U(P_{n_k}, f)\}$, which converges to $H$.
\\Since $\{U(P_{n_k}, f)-L(P_{n_k},f)\}$ is also a subsequence of $\{U(P_n, f)-L(P_n, f)\}$, it converges to the same limit as $\{U(P_n, f)-L(P_n, f)\}$. Hence:
\begin{flalign*}	
        &\lim\limits_{k\to \oo} \big(U(P_{n_k}, f)-L(P_{n_k},f)\big)=0\\
        \imply&\lim\limits_{k \to \oo}\big(U(P_{n_k}, f)-(U(P_{n_k}, f)-L(P_{n_k},f))\big)=H-0 \\
        \imply&\lim\limits_{k\to \oo}L(P_{n_k}, f)=H
\end{flalign*}
This shows $\{{P_{n_k}}\}_{k=1}^{+\oo}$ is a sequence such that:
$$\lim\limits_{n\to \oo} U(P_{n_k}, f)=\lim\limits_{n\to \oo}L(P_{n_k}, f)$$
\paragraph{(iii)$\imply$ (iv)}~{} 
\\\\Denote $m:=\inf_{x\in [a,b]} f(x)$.
Here we quote a result from lecture note (page 169, 4.2.3):
$$L(P, f-m)=L(P,f)-m(b-a)$$
$$U(P, f-m)=U(P,f)-m(b-a)$$
Given that there exists a sequence of partitions $\{P_n\}$ such that $\lim\limits_{n\to \oo} U(P_n,f)=\lim\limits_{n\to \oo} L(P_n,f)$, suppose $\lim\limits_{n\to \oo} U(P_n,f)=I$, then:
 $$\lim\limits_{n\to\infty} L(P_n, f-m)=\lim\limits_{n\to \infty}\big(L(P_n,f)-m(b-a)\big)=I-m(b-a)$$
 $$\lim\limits_{n\to\infty} U(P_n, f-m)=\lim\limits_{n\to \infty}\big(U(P_n,f)-m(b-a)\big)=I-m(b-a)$$
Denote $\W:=G_{[a,b]}(f-m)$
\\Since $f(x)-m$ is a non-negative function on interval $[a, b]$, $U(P_n, f-m)$ is an outer Jordan measure and $L(P_n, f-m)$ is an inner Jordan measure.
\\Thus by the fact that $\{U(P_n,f-m)\}$ and $\{L(P_n, f-m)\}$ converges to the same limit, we conclude that $\W$ is Jordan measurable with $\mu(\W)=I-m(b-a)$
\\Denote $\W_1:=G^+(f-m)+m$, then by translational invariance, $\W_1$ is also Jordan measurable with $\mu (\W_1)=I-m(b-a)$
\\Next we consider $\W_1 \cap ([a,b]\x[0, \sup_{[a, b]}f])$:
\\For any $(x, y) \in G_{[a,b]}^+(f)$: 
$$0<y<f(x)\imply -m<y-m<f(x)-m\imply (x,y-m)\in G^+(f-m)\imply (x, y)\in \W_1$$ 
And:
$$(x,y)\in ([a,b]\times[0, \sup_{[a,b]}f])$$
thus $(x, y)\in \W_1 \cap ([a,b]\x[0, \sup_{[a, b]}f])$
\\Moreover, for any $(x, y)$ in $\W_1 \cap ([a,b]\x[0, \sup_{[a, b]}f])$, 
$$0\le y \text{ and } y-m\le f(x)-m \imply 0\le y\le f(x)$$
thus $(x, y)\in G_{[a,b]}^+(f)$
\\Hence $\W_1 \cap ([a,b]\x[0, \sup_{[a, b]}f])=G_{[a,b]}^+(f)$
\\Similarly, $\W_1 \cap ([a,b]\x[m, 0])=G_{[a,b]}^-(f)$
\\This shows both $G_{[a,b]}^+(f)$ and $G_{[a,b]}^-(f)$ are Jordan measurable, hence f is Riemann integrable on $[a,b]$.
\paragraph{(iv)$\imply$ (v)}~{}
\\\\Denote $m:=\inf_{x\in [a,b]} f(x)$.
\\Given That $f$ is Riemann integrable on $[a,b]$, both $\big(G_{[a,b]}^+(f)\big)$ and $\big(G_{[a,b]}^-(f)\big)$ are also Riemann Integrable
\\Denote: 
$$\W_1=G_{[a,b]}^+(f),\quad \W_2=G_{[a,b]}^-(f),\quad \W_3=G_{[a,b]}^+(f-m)$$
$$\W_4=\{(x,y)|a\le x\le b, m\le y\le f(x) \}$$
\\Hence, $\W_4=\W_1\cup \big([a,b]\x [m,0)\big)\backslash\W_2$
\\Hence $\W_4$ is also Jordan measurable.
\\Then, by translational invariance, $\W_3=\W_4-m$ is also Jordan measurable.
\\Thus, we can find two sequences of inner and outer Jordan measure $\{S_n\}$ and $\{T_n\}$, where 
$$\lim\limits_{n\to \oo}A(S_n)=\mu (\W_3)=\lim\limits_{n\to \oo}A(T_n)$$
Consequently, we can find two sequences of Lower and Upper Darboux sums $\{L(P_n, f-m)\}$ and $\{U(P_n,f-m)\}$ such that: 
$$\lim\limits_{n\to\infty} L(P_n, f-m)=\mu (\W_3)=\lim\limits_{n\to \infty}U(P_n, f-m)$$
By the definition of Darboux sums:
$$U(P_n, f-m)=\sum_{i=1}^{n}\sup\limits_{[x_{i-1}, x_i]}(f-m)\cdot (x_i-x_{i-1})$$
Since we have
$$\sup\limits_{[x_{i-1}, x_{i}]}(f-m)=\sup\limits_{[x_{i-1}, x_{i}]}(f)-m$$
hence:
\begin{flalign*}
	&U(P_n, f-m)\\
	=&\sum_{i=1}^{n}\sup\limits_{[x_{i-1}, x_i]}(f-m)\cdot (x_i-x_{i-1})\\
	=&\sum_{i=1}^{n}\sup\limits_{[x_{i-1}, x_i]}(f)\cdot (x_i-x_{i-1})-m\cdot \sum_{i=1}^{n}(x_{i-1},x_i)\\
	=&U(P_n, f)-m(b-a)
\end{flalign*}
Similarly
$$L(P_n, f-m)= L(P_n,f)-m(b-a)$$
Hence
$$\lim\limits_{n\to \oo}(L(P_n,f)-m(b-a))=\lim\limits_{n\to \oo}(U(P_n, f)-m(b-a))$$
Hence
$$\lim\limits_{n\to \oo}(L(P_n,f))=\lim\limits_{n\to \oo}U(P_n, f)$$
Since $\forall n \in \N $, we have $U(P,f-m)\ge L(P_n,f-m)$ for any partition $P$
\\Hence by comparison rules, $\lim\limits_{n\to \oo}L(P_n, f-m)\le U(P, f-m)$ for any partition $P$.
\\Hence $\lim\limits_{n\to \oo} U(P_n, f-m)=\lim\limits_{n\to \oo}L(P_n, f-m)$ is a lower bound for set $\{U(P, f-m)|\text{P is a partition of [a, b]}\}$
\\Moreover, for any $M<\lim\limits_{n\to \oo} U(P_n, f-m)$, by order rules we can find $N\in \N$ s.t. $\forall n\ge N, U(P_n,f-m)>M$, which says $M$ is not an upper bound
\\Hence, 
\begin{flalign*}
\lim\limits_{n\to \oo}U(P_n,f-m)&=\inf\{U(P, f-m)|\text{P is a partition of [a,b]}\}\\
\lim\limits_{n\to \oo}\big(U(P_n,f)-m(b-a)\big)&=\inf\{U(P, f)-m(b-a)|\text{P is a partition of [a,b]}\\
\lim\limits_{n\to \oo}U(P_n,f)-m(b-a)&=\inf\{U(P, f)|\text{P is a partition of [a,b]}\}-m(b-a)\\
\lim\limits_{n\to \oo}U(P_n,f)&=\inf\{U(P, f)|\text{P is a partition of [a,b]}\}
\end{flalign*} 
Similarly, 
$$\lim\limits_{n\to \oo}L(P_n,f)=\sup\{L(P, f)|\text{P is a partition of [a,b]}\}$$
Thus,  $\sup\{L(P, f)|\text{P is a partition of [a,b]}\}=\inf\{U(P, f)|\text{P is a partition of [a,b]}\}$
$$\imply \underline{\displaystyle\int_{a}^{b}}f(x)dx=\overline{\displaystyle\int_{a}^{b}}f(x)dx$$
\paragraph{(v) $\imply$ (i)}~{}
\\\\
Given that $\underline{\displaystyle\int_{a}^{b}}f(x)dx=\overline{\displaystyle\int_{a}^{b}}f(x)dx$,
\\by definition: $\sup\{L(P, f)|\text{P is a partition of [a,b]}\}=\inf\{U(P, f)|\text{P is a partition of [a,b]}\}$
\\Choose $\varepsilon>0$, by the property of supremum and infimum:
$$\exists U(P_1, f) \in \{U(P, f)|\text{P is a partition of [a,b]}\} s.t. 
U(P_1,f)-\inf\{U(P, f)|\text{P is a partition of [a,b]}\}<\frac{\varepsilon}{2}$$
$$\exists L(P_2, f) \in \{L(P, f)|\text{P is a partition of [a,b]}\} s.t. 
\sup\{L(P, f)|\text{P is a partition of [a,b]}\}-L(P_2, f)<\frac{\varepsilon}{2}$$
\\Moreover, we argue that for partition $P_3:=P_1\cup P_2$, $U(P_1, f)\ge U(P_3, f)$ and $L(P_1, f)\le L(P_3, f)$.
\\First we'll justify that for a partition $P'$ produced by plug in a point into the original partition $P$, we have
\begin{flalign*}
	&U(P', f)\le U(P, f)\\
	&L(P', f)\ge L(P, f)
\end{flalign*} 
Suppose that $P'=P\cup\{c\}$ where $c\in (x_k, x_{k+1})$\\
\\then, 
\begin{flalign*}
	U(P, f)&=\sum_{i=1}^{n}\sup\limits_{[x_{i-1}, x_i]}f\cdot (x_i-x_{i-1})\\
	U(P',f)&=\sum_{i=1}^{k}\sup\limits_{[x_{i-1}, x_i]}f\cdot (x_i-x_{i-1})+\sum_{k+1}^{n}\sup\limits_{[x_{i-1}, x_i]}f\cdot (x_i-x_{i-1})+\sup\limits_{[x_{k}, c]}f\cdot (c-x_k)+\sup\limits_{[c, x_{k+1}]}f\cdot (x_{k+1}-c)
\end{flalign*}
Since $[x_k, c]\in[x_k, x_{k+1}]$ and $[c, x_{k+1}]\in[x_k, x_{k+1}]$, there is $\sup\limits_{[x_k, x_{k+1}]}f \ge \sup\limits_{[x_k, c]}f$ and $\sup\limits_{[x_k, x_{k+1}]}f\ge \sup\limits_{[c, x_{k+1}]}f$ 
\\hence,
\begin{flalign*}
	U(P, f)-U(P', f)&=\sup\limits_{[x_k, x_{k+1}]}f\cdot (x_{k+1}-x_{k})-\sup\limits_{[x_k, c]}f \cdot (c-x_k)-\sup\limits_{[x_k, x_{k+1}]}f\cdot (x_{k+1}-c) \\
	                &\ge\sup\limits_{[x_k, x_{k+1}]}f\cdot (x_{k+1}-x_{k})-\sup\limits_{[x_k, x_{k+1}]}f\cdot (x_{k+1}-c+c-x_{k})\\
	                &=0
\end{flalign*}
and hence $U(P,f)\ge U(P', f)$
\\Similarly,
\begin{flalign*}
	L(P, f)-L(P', f)&=\inf\limits_{[x_k, x_{k+1}]}f\cdot (x_{k+1}-x_{k})-\inf\limits_{[x_k, c]}f \cdot (c-x_k)-\inf\limits_{[x_k, x_{k+1}]}f\cdot (x_{k+1}-c) \\
	&\le\inf\limits_{[x_k, x_{k+1}]}f\cdot (x_{k+1}-x_{k})-\inf\limits_{[x_k, x_{k+1}]}f\cdot (x_{k+1}-c+c-x_{k})\\
	&=0
\end{flalign*}
and hence, $L(P ,f)\le L(P', f)$
\\Since $P_3$ is produced by plugging in finitely many points in to $P_1$ or $P_2$, by induction we have
$$U(P_1, f)\ge U(P_3, f) \text{ and } L(P_1, f)\le L(P_3, f)$$
consequently,
\begin{flalign*}
U(P_3,f)-L(P_3, f)
\le&U(P_1, f)-L(P_2, f)\\
=&U(P_1,f)-\inf\{U(P, f)|\text{P is a partition of [a,b]}\}\\&\;-\big(L(P_2,f)-\sup\{L(P, f)|\text{P is a partition of  [a,b]}\}\big)\\
<&\frac{\varepsilon}{2}+\frac{\varepsilon}{2}\\
=&\varepsilon	
\end{flalign*}
This shows that we've found a partition $P_3$ where
$$U(P_3, f)-L(P_3, f)<\e$$
\\and hence completes the proof.

\end{document}