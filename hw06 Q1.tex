\documentclass{article}
\usepackage[fontsize=11pt]{fontsize}
\usepackage{graphicx,amsmath,amsfonts,mathabx, amsthm, amssymb, bm, hyperref, mathrsfs}
\usepackage{fontspec}
\usepackage{geometry}
\usepackage{boondox-cal}
\usepackage{color}
\geometry{a4paper,scale=0.75}
\setmainfont{Times New Roman}
\title{ MATH1024 Problem Set 7}
\author{ Group member:}
\date{May 2024}
\begin{document}
\setlength {\parindent} {0pt}
\maketitle



\paragraph[short]{1.(a)}
{By Taylor's expansion, we have 
$$\log(1+x)=x-\frac{x^2}{2}+\frac{x^3}{3}-\frac{x^4}{4}+\cdots$$
Thus
$$\log(1+x)-x=\frac{x^2}{2}-\frac{x^3}{3}+\frac{x^4}{4}-\cdots=\sum_{i=2}^{\infty}(
\frac{(-1)^{i-1}x^i}{i})$$
Since $\forall i\in\mathbb{N}$,$ \frac{(-1)^{i-1}x^i}{i}<|x^i|$ and $x\in(-1,1)$, 
we have
$$\sum_{i=2}^{\infty}(
    \frac{(-1)^{i-1}x^i}{i})\leq
    \sum_{i=2}^{\infty}|x^i|=\frac{{|x|}^2}{1-|x|}
    $$
Thus we finish the first inequality
$$
\log(1+x)-x\leq\frac{{|x|}^2}{1-|x|}$$
For the second equality:
By Taylor's expansion and Lagrange remainder, we have
$$\log(1+x)=x-\frac{x^2}{2}+\frac{x^3}{3}-\frac{x^4}{4}+\cdots+\frac{(-1)^{n-1}x^n}{n}+R_n(x)$$
where $R_n(x)=\frac{(-1)^{n-1}x^{n+1}}{(n+1)(1+c)^{n+1}}$ for some $c\in(-1,1)$
and 
$$\log(1-x)=
-x-\frac{x^2}{2}-\frac{x^3}{3}-\frac{x^4}{4}-\cdots-\frac{x^n}{n}+R_n(-x)$$
where $R_n(-x)=
\frac{(-1)^{n}x^{n+1}}{(n+1)(1+c)^{n+1}}$ for some $c\in(-1,1)$\\
Thus $$R_2(-x)=\frac{x^3}{3(1+c)^3}=O(x^3)$$ 
$$R_2(x)=\frac{x^3}{3(1+c)^3}=O(x^3)$$
Thus 
$$\log(1+x)-\log(1-x)+x=2x+O(x^3)$$
Namely
$$\log\frac{1+x}{1-x}=2x+O(x^3)$$
Thus we finish the second equality.

}

\paragraph[short]{1.(b)}
{
    Firstly we have
    $$  \int_{\frac{1}{2}}^{n+\frac{1}{2}}\log t dt=\sum_{k=1}^{n}
    \int_{k-\frac{1}{2}}^{k+\frac{1}{2}}\log t dt$$
    $$\log n!=\log 1+\log2 +\cdots+\log n=\sum_{k=1}^{n}\log k$$
    Thus by combining the above two equations, we have
\begin{align*}
\log(n!)&=\sum_{k=1}^{n}\log k\\
&=\int_{\frac{1}{2}}^{n+\frac{1}{2}}\log t dt-
\sum_{k=1}^{n}\int_{k-\frac{1}{2}}^{k+\frac{1}{2}}+\log t dt+\sum_{k=1}^{n}\log k\\
&=\int_{\frac{1}{2}}^{n+\frac{1}{2}}
\log t dt+\sum_{k=1}^{n}\left(\log k-\int_{k-\frac{1}{2}}^{k+\frac{1}{2}}
\log t dt\right)
\end{align*}
}

\paragraph[short]{1.(c)}
{Firstly we have:
$$  \int_{}^{}\log t dt=t\log t-t+C$$
Thus $$\int_{\frac{1}{2}}^{n+\frac{1}{2}}\log t dt=
(n+\frac{1}{2})\log(n+\frac{1}{2})+\frac{1}{2}\log2-n$$
Thus
$$\int_{\frac{1}{2}}^{n+\frac{1}{2}}\log t dt-\left((n+\frac{1}{2})\log n
-n+\frac{1+\log 2}{2}+O(\frac{1}{n})\right)=(n+\frac{1}{2})\log (1+\frac{1}{2n})-\frac{1}{2}$$
Then by Taylor's expansion and Lagrange remainder, we have
\begin{align*}
    (n+\frac{1}{2})\log (1+\frac{1}{2n})-\frac{1}{2}&=(n+\frac{1}{2})(\frac{1}{2n}+O(\frac{1}{4n^2}))-\frac{1}{2}\\
    &=\frac{1}{2}+O(\frac{1}{n})+\frac{1}{4n}+O(\frac{1}{n^2})-\frac{1}{2}\\
    &=O(\frac{1}{n})
\end{align*}
Thus we have
$$\int_{\frac{1}{2}}^{n+\frac{1}{2}}\log t dt=
(n+\frac{1}{2})\log(n)+\frac{1+\log 2}{2}-n+O(\frac{1}{n})$$
}
\paragraph[short]{1.(d)}
{
  Firstly, by 
  $$\int \log t dt=t\log t-t+C$$
  we have
  $$\int_{k}^{k+\frac{1}{2}}\log t dt =
    (k+\frac{1}{2})\log(k+\frac{1}{2})-k\log k-\frac{1}{2}$$
and
$$\int_{k-\frac{1}{2}}^{k}\log t dt =
    k\log k-(k-\frac{1}{2})\log(k-\frac{1}{2})-\frac{1}{2}$$
Thus   
\begin{align*}
    \left|\frac{1}{2}\log k-\int_{k}^{k+\frac{1}{2}}\log t dt\right|&=\left|\frac{1}{2}\log k-(k+\frac{1}{2})\log(k+\frac{1}{2})+k\log k+\frac{1}{2}\right|\\
    &=(k+\frac{1}{2})\log(1+\frac{1}{2k})-\frac{1}{2}\\
    &=(k+\frac{1}{2})(\frac{1}{2k}+\frac{1}{8k^2}+O(\frac{1}{k^3}))-\frac{1}{2}\\
    &=\frac{1}{8k}+O(\frac{1}{k^2})
\end{align*}
Similarly, we have
\begin{align*}
    \left|\frac{1}{2}\log k-\int_{k-\frac{1}{2}}^{k}\log t dt\right|&=\left|\frac{1}{2}\log k-k\log k+(k-\frac{1}{2})\log(k-\frac{1}{2})+\frac{1}{2}\right|\\
    &=-(k-\frac{1}{2})\log(1-\frac{1}{2k})+\frac{1}{2}\\
    &=-(k-\frac{1}{2})(-\frac{1}{2k}+\frac{1}{8k^2}+O(\frac{1}{k^3}))+\frac{1}{2}\\
    &=\frac{1}{8k}+O(\frac{1}{k^2})
\end{align*}
Then combining the above two equations, we have
\begin{align*}
    \log k-\int_{k-\frac{1}{2}}^{k+\frac{1}{2}}\log t dt&= \frac{1}{2}\log k-\int_{k}^{k+\frac{1}{2}}\log t dt+\frac{1}{2}\log k-\int_{k-\frac{1}{2}}^{k}\log t dt\\
    &=-\left|\frac{1}{2}\log k-\int_{k}^{k+\frac{1}{2}}\log t dt\right|+\left|\frac{1}{2}\log k-\int_{k-\frac{1}{2}}^{k}\log t dt\right|\\
    &=(\frac{1}{8k}+O(\frac{1}{k^2}))-(\frac{1}{8k}+O(\frac{1}{k^2}))\\
    &=O(\frac{1}{k^2})
\end{align*}
Then trivially we have the convergence of the series:
$$\left|\log k-\int_{k-\frac{1}{2}}^{k+\frac{1}{2}}\log t dt \right| =O(\frac{1}{k^2})$$
Then by the comparison test, we have
$$\sum_{k=1}^{\infty}\left|\log k-\int_{k-\frac{1}{2}}^{k+\frac{1}{2}}\log t dt \right|$$
 
Thus by the absulute convergence test, we have the convergence of the series:
$$\sum_{k=1}^{\infty}\left(\log k-\int_{k-\frac{1}{2}}^{k+\frac{1}{2}}\log t dt \right)$$
}

\paragraph[short]{1.(e)}
{
By the above analysis, we have
$$\int_{\frac{1}{2}}^{n+\frac{1}{2}}\log t dt=
(n+\frac{1}{2})\log(n)+\frac{1+\log 2}{2}-n+O(\frac{1}{n})$$
\begin{align*}
    \log n!&=\int_{\frac{1}{2}}^{n+\frac{1}{2}}\log t dt+\sum_{k=1}^{n}\left(\log k-\int_{k-\frac{1}{2}}^{k+\frac{1}{2}}\log t dt\right)\\
    &=\int_{\frac{1}{2}}^{n+\frac{1}{2}}\log t dt+\sum_{k=1}^{n}\left(\log k-\int_{k-\frac{1}{2}}^{k+\frac{1}{2}}\log t dt\right)\\
    &=(n+\frac{1}{2})\log n+\frac{1+\log 2}{2}-n+O(\frac{1}{n})+\sum_{k=1}^{n}\left(\log k-\int_{k-\frac{1}{2}}^{k+\frac{1}{2}}\log t dt\right)\\
\end{align*}
Thus
\begin{align*}
    \log n!-(n+\frac{1}{2})\log n+n&=\frac{1+\log 2}{2}+O(\frac{1}{n})+\sum_{k=1}^{n}\left(\log k-\int_{k-\frac{1}{2}}^{k+\frac{1}{2}}\log t dt\right)\\
    &=\frac{1+\log 2}{2}+\sum_{k=1}^{n}\left(\log k-\int_{k-\frac{1}{2}}^{k+\frac{1}{2}}\log t dt\right)+O(\frac{1}{n})\\
\end{align*}
Thus by the convergence of 
$$\sum_{k=1}^{\infty}\left(\log k-\int_{k-\frac{1}{2}}^{k+\frac{1}{2}}\log t dt \right),$$
$\left(\log n!-(n+\frac{1}{2})\log n+n\right)$
converges to a finite value.
Thus
$$\lim_{n\to\infty}\log \left(\frac{n!e^n}{n^{n+\frac{1}{2}}}\right)$$
exists.
By the continuity of the exponential function, we have
$$\lim_{n\to\infty}\frac{n!e^n}{n^{n+\frac{1}{2}}}=:L$$
exists.
}

\paragraph[short]{1.(f)}
{
We can prove them by induction.\\
For $n=0$, we have
$$I_0=\int_{0}^{\frac{\pi}{2}}1 dt=\frac{\pi}{2}=C^0_0\frac{\pi}{2^1}$$
$$I_1=\int_{0}^{\frac{\pi}{2}}\sin t dt=1=\frac{2^0(0!)^2}{(1)!}$$
Then if we have:
$$I_{2n}=C^{2n}_n\frac{\pi}{2^{2n+1}}$$
$$I_{2n+1}=\frac{2^{2n}(n!)^2}{(2n+1)!}$$
Then:
\begin{align*}
    I_{2n+2}&=\int_{0}^{\frac{\pi}{2}}\sin^{2n+2}t dt\\
    &=\int_{0}^{\frac{\pi}{2}}\sin^{2n+1}t\sin t dt\\
    &=\int_{0}^{\frac{\pi}{2}}\sin^{2n+1}t d(-\cos t)\\
    &=\left.\sin^{2n+1}t(-\cos t)\right|_{0}^{\frac{\pi}{2}}-\int_{0}^{\frac{\pi}{2}}(-\cos t)d\sin^{2n+1}t\\
    &=(2n+1)\int_{0}^{\frac{\pi}{2}}\sin^{2n}t\cos^2t dt\\
    &=(2n+1)\int_{0}^{\frac{\pi}{2}}\sin^{2n}t(1-\sin^2t) dt\\
    &=(2n+1)\int_{0}^{\frac{\pi}{2}}\sin^{2n}t dt-(2n+1)\int_{0}^{\frac{\pi}{2}}\sin^{2n+2}t dt\\
    &=(2n+1)I_{2n}-(2n+1)I_{2n+2}
\end{align*}
Thus
$$I_{2n+2}=\frac{2n+1}{2n+2}I_{2n}=C^{2(n+1)}_{n+1}\frac{\pi}{2^{2(n+1)+1}}$$
\begin{align*}
    I_{2n+3}&=\int_{0}^{\frac{\pi}{2}}\sin^{2n+3}t dt\\
    &=\int_{0}^{\frac{\pi}{2}}\sin^{2n+2}t\sin t dt\\
    &=\int_{0}^{\frac{\pi}{2}}\sin^{2n+2}t d(-\cos t)\\
    &=\left.\sin^{2n+2}t(-\cos t)\right|_{0}^{\frac{\pi}{2}}-\int_{0}^{\frac{\pi}{2}}(-\cos t)d\sin^{2n+2}t\\
    &=(2n+2)\int_{0}^{\frac{\pi}{2}}\sin^{2n+1}t\cos^2t dt\\
    &=(2n+2)\int_{0}^{\frac{\pi}{2}}\sin^{2n+1}t(1-\sin^2t) dt\\
    &=(2n+2)\int_{0}^{\frac{\pi}{2}}\sin^{2n+1}t dt-(2n+2)\int_{0}^{\frac{\pi}{2}}\sin^{2n+3}t dt\\
    &=(2n+2)I_{2n+1}-(2n+2)I_{2n+3}
\end{align*}
Thus
$$I_{2n+3}=\frac{2n+2}{2n+3}I_{2n+1}=\frac{2^{2(n+1)}(n+1)!^2}{(2n+3)!}$$
Thus by induction, we have
$$I_{2n}=
C^{2n}_n\frac{\pi}{2^{2n+1}}$$
$$I_{2n+1}=\frac{2^{2n}(n!)^2}{(2n+1)!}$$
}
\paragraph[short]{1.(g)}
{
Firstly, to prove that
$$\lim_{n\to\infty}\frac{I_{2n}}{I_{2n+1}}=1,$$
It suffices to prove that the existence of the limit of the sequence:
$$\lim_{n\to\infty}I_n=:M.$$
As we know, the sequence $\{I_n\}$ is bounded below by $0$\
Also by $\sin^{n+1} t\leq \sin^{n}t$, we have
$$I_{n+1}\leq I_{n}$$
Thus the sequence $\{I_n\}$ is monotonic decreasing and bounded below by $0$.
Thus the limit of the sequence exists.\\
Thus:
$$\lim_{n\to\infty}\frac{I_{2n}}{I_{2n+1}}=1$$
Thus 
$$\lim_{n\to\infty}\frac{C^{2n}_n\frac{\pi}{2^{2n+1}}}{\frac{2^{2n}(n!)^2}{(2n+1)!}}=1$$
In other words,
$$\lim_{n\to\infty}\frac{(n!)^42^{4n}}{(2n)!(2n+1)!}=\frac{\pi}{2}$$

Also by the definition of $L$:
$$L=\sqrt{2\pi}\iff \lim_{n\to\infty}\frac{(n!)^2e^{2n}}{4\cdot n^{2n+1}}=\frac{\pi}{2}$$
Thus
\begin{align*}
L=\sqrt{2\pi}&\iff \lim_{n\to\infty}\frac{(n!)^2e^{2n}}{4\cdot n^{2n+1}}=\lim_{n\to\infty}\frac{(n!)^42^{4n}}{(2n)!(2n+1)!}\\
&\iff \lim_{n\to\infty}\frac{4(n!)^2((\frac{16}{e^2})^n\cdot n^{2n+1})}{(2n)!(2n+1)!}=1\\
&\iff \lim_{n\to\infty}\frac{
    4(\frac{L\cdot n^{n+\frac{1}{2}}}{e^n})^2(\frac{16}{e^2})^n\cdot n^{2n+1}
}{
    \frac{L\cdot(2n)^{2n+\frac{1}{2}}}{e^{2n}}\cdot\frac{L\cdot(2n)^{2n+\frac{1}{2}}}{e^{2n}}\cdot(2n+1)
}=1\\
&\iff \lim_{n\to\infty}\frac{
    4\cdot n^{4n+2}(16)^n
}{
    2^{4n+1}\cdot n^{4n+1}\cdot(2n+1)
}=1\\
&\iff \lim_{n\to\infty}\frac{2n}{2n+1}=1
\end{align*}
It's trivial to prove that
$$\lim_{n\to\infty}\frac{2n}{2n+1}=1$$
Thus we finish the proof.
}



 \end{document}
