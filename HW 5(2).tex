\documentclass{article}
\usepackage{amsfonts, amsmath, amsthm, amssymb, color, float, geometry, graphicx, mathrsfs, setspace, subfigure, yhmath}
\title{\LaTeX{} Answer to Problem Set \#5}
\author{AO Yuzhuo, JIN Zhibo, LYU Changlai\thanks{names listed in alphabetical order}}
\date{\today}
\everymath{\displaystyle}

\newtheorem{Thm}{Theorem}[section]
\newtheorem{Lem}[Thm]{Lemma}
\newtheorem{Prop}[Thm]{Proposition}
\newtheorem{Cor}[Thm]{Corollary}
\newtheorem{Rem}[Thm]{Remark}
\newtheorem{Def}[Thm]{Definition}
\def\sep{\vspace{1cm}\hrule\vspace{1cm}}
\def\tab{\;\;\;\;\;\;}
\def\R{\mathbb{R}}
\def\Q{\mathbb{Q}}
\def\N{\mathbb{N}}
\def\C{\mathbb{C}}
\def\Z{\mathbb{Z}}
\def\a{\alpha}
\def\b{\beta}
\def\c{\gamma}
\def\d{\delta}
\def\e{\epsilon}
\def\p{\partial}
\def\h{\theta}
\def\w{\omega}
\def\bu{\mathbf{u}}
\def\bv{\mathbf{v}}
\def\iff{\Longleftrightarrow}
\def\oif{\Rightarrow}
\def\to{\rightarrow}
\def\inj{\hookrightarrow}
\def\surj{\twoheadrightarrow}
\def\imply{\Longrightarrow}
\def\x{\times}
\def\<{\langle}
\def\>{\rangle}
\def\oo{\infty}
\def\normal{\triangleleft}
\def\h{\hspace*{0.5cm}}
\def\limsupa{\limsup\limits_{n\to\infty}{a_n}}
\def\limsupb{\limsup\limits_{n\to\infty}{b_n}}
\def\liminfa{\liminf\limits_{n\to\infty}{a_n}}
\def\liminfb{\liminf\limits_{n\to\infty}{b_n}}
\def\limsupn#1{\limsup\limits_{n\to\infty}\bigg({#1}\bigg)}
\def\liminfn#1{\liminf\limits_{n\to\infty}\bigg({#1}\bigg)}
\def\LIM{\text{LIM}}
\def\seqa{\{a_n\}}
\def\seqb{\{b_n\}}
\def\sseqa{\{a_{n_k}\}}
\def\st{\text{ s.t. } }
\def\when{\text{ when }}

\begin{document}
\maketitle

\section{}
Consider the right sum and the left sum,
as we know ,$\forall n$,
$$L_n<\int_{2}^{3}e^{-x^2}<R_n.$$
Take $n=10000$, then we have $R_n=0.0041268671689888405,L_n=0.004125047946080374$, thus
$$0.004125047946080374<\int_{2}^{3}e^{-x^2}<0.0041268671689888405.$$
Thus the $'3-sig. fig.'$ approximation should be $0.00413$.\\
$\therefore$ We need to find the appropriate partition s.t. the $'3-sig. fig.'$ approximations of the 
following methods should be $0.00413$

\paragraph{Right Sum}
$$\begin{aligned}
    &f'(x) = -2xe^{-x^2} \\
    &\forall x \in [2,3] ,\, f''(x)>0 \implies f'(x) \text{ monotonely increases on } [2,3] \\
    &\sup_{[2,3]} \left|f'(x)\right| = \left|f'(2)\right| < 0.0733 =: m_1
\end{aligned}$$
To keep the ‘3 sig. fig.’ approximation accurate, it is sufficient to let $\left| \int_a^b f(x)dx - R_n \right| < \frac{1}{1000}R_n$.
Through simple trials, we estimated that $0.004<R_n<0.005$.
\\\\
Therefore, $\left| \int_a^b f(x)dx - R_n \right| < 4\times 10^{-6} \imply \frac{m_1}{2n} < 4\times 10^{-6} \imply n \ge (   )$ suffices our need.
\\\\
However, to my preference, I chose $n=(   )$. :)

\paragraph{Trapezoidal Rule}
$$\begin{aligned}
    &f''(x) = (4x^2-2)e^{-x^2} ,\, f'''(x) = (-8x^3+12x)e^{-x^2} \\
    &\forall x \in [2,3] ,\, f'''(x)<0 \implies f''(x) \text{ monotonely decreases on } [2,3] \\
    &\sup_{[2,3]} \left|f''(x)\right| = \left|f''(2)\right| < 0.2565 =: m_2
\end{aligned}$$
Therefore, $\left| \int_a^b f(x)dx - T_n \right| < 4\times 10^{-6} \Longleftarrow \frac{m_2}{12n^2} < 4\times 10^{-6} \Longleftarrow n \ge 199$ suffices our need.
\\\\
However, to play safe, I chose $n=200$. :)

\paragraph{Simpson's Rule}
Claim: $n=200$ suffices this approximation. Then we justify the choice of this $n$:\\
We know that $$\int_{2}^{3}e^{-x^2}dx<\int_{2}^{3}1dx=1$$
Then assume that $10^{-m}\le \int_{2}^{3}e^{-x^2}dx<10^{-m+1}$ where $m\in \N$, hence we can write $\int_{2}^{3}e^{-x^2}dx$ as the following series:
$$\int_{2}^{3}e^{-x^2}dx=\sum_{k=1}^{+\oo}a_k\cdot 10^{-(m+k)}$$
where $a_k$ is the $m+k$-th digit of the number $\int_{2}^{3}e^{-x^2}dx$ and $a_k\in \N$\\
Then for the sake of accuracy, we need to ensure that the error $E$ satisfies:
$$E=\big|\int_{2}^{3}e^{-x^2}dx-S_n\big|<10^{-(m+4)}\cdot a_5$$
where $S_n$ is the $n$-th partial sum of the Simpson's Rule.\\
Moreover, we have 
$$10^{-(m+4)}\cdot a_5\ge \cdot 10^{-m-4}\ge 10^{-4}\cdot \int_{2}^{3}e^{-x^2}dx$$ 
hence it suffices to bound the error by $10^{-4}\cdot \int_{2}^{3}e^{-x^2}dx$\\ 
thus it is equivalent to find an $n$ such that
$$0.9999\cdot \int_{2}^{3}e^{-x^2}dx\le S_n\le 1.0001\cdot \int_{2}^{3}e^{-x^2}dx$$
$$\imply \frac{S_n}{1.0001} < \int_{2}^{3}e^{-x^2}dx < \frac{S_n}{0.9999}$$
hence it suffices to find n such that
$$E=\big|\int_{2}^{3}e^{-x^2}dx-S_n\big|<\frac{0.0001}{2}S_n$$
hence we need to find $n$ such that
$$\frac{(3-2)^5}{C\cdot n^4}\sup\limits_{[2,3]}|f^{(4)}|<\frac{0.0001}{2}\cdot e^{-9}$$
where $C$ is a constant independent of $n$\\
And we know that $C\ge 1$, and $\sup\limits_{[2, 3]}|f^{(4)}|<1$ hence it suffices to find $n$ such that
$$n^4>2\cdot 10^4\cdot e^9$$
and n= 200 satisfies this inequality.\\


\section{}
\paragraph{(a)}
Since $P_i(x)$ is a uniform partition, denote $\Delta x := \dfrac{b-a}{3n}$:

$$\begin{aligned}
    &\forall i \in \{0,1,2,\cdots,n-1\}, P_i(x) = \frac{(x-x_{3i+1})(x-x_{3i+2})(x-x_{3i+3})}{-6\Delta x^3} f(x_{3i}) + \cdots \\
    &\int_{x_{3i}}^{x_{3i+3}} P_i(x) = -\frac{f(x_{3i})}{6\Delta x^3} \int_{x_{3i}}^{x_{3i+3}} (x-x_{3i+1})(x-x_{3i+2})(x-x_{3i+3}) d\left(x-x_{3i+2}\right) + \cdots - \cdots + \cdots \\
\end{aligned}$$
We consider each part respectively:
$$\begin{aligned} x_{3i} \text{ part: }& -\frac{f(x_{3i})}{6\Delta x^3} \int_{x_{3i}}^{x_{3i+3}} (u^3-u\cdot\Delta x^2) \,du = -\frac{9}{4} \cdot \Delta x \cdot f(x_{3i}) \\
x_{3i+1} \text{ part: }& \frac{f(x_{3i+1})}{6\Delta x^3} \int_{x_{3i}}^{x_{3i+3}} (u+2\Delta x)\cdot u\cdot (u-\Delta x) \,du = \frac{9}{4} \cdot \Delta x \cdot f(x_{3i+1}) \\
x_{3i+2} \text{ part: }& \frac{f(x_{3i+2})}{6\Delta x^3} \int_{x_{3i}}^{x_{3i+3}} (u+\Delta x)\cdot u\cdot (u-2\Delta x) \,du = \frac{9}{4} \cdot \Delta x \cdot f(x_{3i+2}) \\
x_{3i+3} \text{ part: }& \frac{f(x_{3i+3})}{6\Delta x^3} \int_{x_{3i}}^{x_{3i+3}} (u^3-u\cdot\Delta x^2) \,du = \frac{9}{4} \cdot \Delta x \cdot f(x_{3i+3}) \\
\end{aligned}$$
Altogether:
$$\begin{aligned}
    \int_{x_{3i}}^{x_{3i+3}} P_i(x) = \frac{9}{4} \cdot \Delta x \left( -f(x_{3i}) + f(x_{3i+1}) + f(x_{3i+2}) + f(x_{3i+3}) \right)
\end{aligned}$$

\paragraph{(b)}
First lets consider the taylor approximation of $f$ at $x_{3i}$:
$$f(x)=f(x_{3i})+f'(x_{3i})(x-x_{3i})+\frac{f''(x_{3i})}{2}(x-x_{3i})^2+\frac{f^{(3)}(x_{3i})}{3!}(x-x_{3i})^3+\frac{f^{(4)}(h(x))}{4!}(x-x_{3i})^4 $$
note that the last term $\frac{f^{(4)}(h(x))}{4!}(x-x_{3i})^4$ is given by \textit{Lagrange~remainder~theorem} and \\$$h(x)\in[x_{3i},x_{3i}+3\Delta x]\subset[a, b]$$\\
denote $$H_i(x)=f(x_{3i})+f'(x_{3i})(x-x_{3i})+\frac{f''(x_{3i})}{2}(x-x_{3i})^2+\frac{f^{(3)}(x_{3i})}{3!}(x-x_{3i})^3$$
hence
\begin{flalign*}
    f(x_{3i}+\Delta x)&=H_i(x_{3i}+\Delta x)+\frac{f^{(4)}(h(x_{3i}+\Delta x))}{4!}(x-x_{3i})^4\\
    f(x_{3i}+2\Delta x)&=H_i(x_{3i}+2\Delta x)+\frac{f^{(4)}(h(x_{3i}+2\Delta x))}{4!}(x-x_{3i})^4\\
    f(x_{3i}+3\Delta x)&=H_i(x_{3i}+3\Delta x)+\frac{f^{(4)}(h(x_{3i}+3\Delta x))}{4!}(x-x_{3i})^4
\end{flalign*}
then we evaluate $\bigg|\int_{x_{3i}}^{x_{3i}+3\Delta x}P_i(x)dx-\int_{x_{3i}}^{x_{3i}+3\Delta x}H_i(x)dx\bigg|$
Previously we've shown that 
$$\int_{x_{3i}}^{x_{3i}+3\Delta x}P_i(x)dx=\frac{3\Delta x}{8}\left(f(x_{3i})+3f(x_{3i}+\Delta x)+3f(x_{3i}+2\Delta x)+f(x_{3i}+3\Delta x)\right)$$
thus we substitute $f(x)$ with $H_i(x)+\frac{f^{(4)}(h(x))}{4!} (x-x_i)^4$ and evaluate the integral:
\begin{flalign*}
    \;&\int_{x_{3i}}^{x_{3i}+3\Delta x}H_i(x)dx\\
    =&\Delta x\cdot \frac{3}{8}\cdot (f(x_{3i})+3f(x_{3i}+\Delta x)+3f(x_{3i}+2\Delta x)+f(x_{3i}+3\Delta x))\\
    =&\Delta x\cdot \frac{3}{8}\cdot \bigg(8\cdot f(x_{3i})+12\Delta x\cdot f'(x_{3i})+12(\Delta x)^2\cdot f''(x_{3i})+9(\Delta x)^3\cdot f^{(3)}(x_{3i})\\    
    \;&+\frac{1}{24}\cdot (\Delta x)^4\cdot \left(3f^{(4)}\left(h(x_{3i}+\Delta x)\right)+48f^{(4)}\left(h(x_{3i}+2\Delta x)\right)+81f^{(4)}\left(h(x_{3i}+3\Delta x)\right)\right) \bigg)\\
    =&3\Delta x\cdot f(x_{3i})+\frac{9}{2} (\Delta x)^2\cdot f'(x_{3i})+\frac{9}{2} (\Delta x)^3\cdot f''(x_{3i})+\frac{27}{8}(\Delta x)^4\cdot f^{(3)}(x_{3i})\\
    \;&+\frac{3}{64}(\Delta x)^5\cdot \left(f^{(4)}(h(x_{3i}+\Delta x))+16f^{(4)}(h(x_{3i}+2\Delta x))+27f^{(4)}(h(x_{3i}+3\Delta x))\right)\\
\end{flalign*}
and integrate $H_i(x)$ over $[x_{3i},x_{3i}+3\Delta x]$ we have:
$$\int_{x_{3i}}^{x_{3i}+3\Delta x}H_i(x)dx=3\Delta x\cdot f(x_{3i})+\frac{9}{2} (\Delta x)^2\cdot f'(x_{3i})+\frac{9}{2} (\Delta x)^3\cdot f''(x_{3i})+\frac{27}{8}(\Delta x)^4\cdot f^{(3)}(x_{3i})$$
hence:
\begin{flalign*}
    \;&\bigg|\int_{x_{3i}}^{x_{3i}+3\Delta x}P_i(x)dx-\int_{x_{3i}}^{x_{3i}+3\Delta x}H_i(x)dx\bigg|\\
    =&\frac{3}{64}(\Delta x)^5\cdot \left(f^{(4)}(h(x_{3i}+\Delta x))+16f^{(4)}(h(x_{3i}+2\Delta x))+27f^{(4)}(h(x_{3i}+3\Delta x))\right)\\
    \le& \sup\limits_{[a, b]}|f^{(4)}|\cdot \frac{3}{64}(\Delta x)^5\cdot (1+16+27)\\
    =&\sup\limits_{[a, b]}|f^{(4)}|\cdot \frac{33}{16}(\Delta x)^5
\end{flalign*}
and:
\begin{flalign*}
    \;&\left|\int_{x_{3i}}^{x_{3i}+3\Delta x}f(x)dx-\int_{x_{3i}}^{x_{3i}+3\Delta x}H_i(x)dx\right|\\
    =&\left|\int_{x_{3i}}^{x_{3i}+3\Delta x}\frac{f^{(4)}(h(x))}{4!}(x-x_{3i})^4dx\right|\\
    \le&\int_{x_{3i}}^{x_{3i}+3\Delta x}\left|\frac{f^{(4)}(h(x))}{4!}(x-x_{3i})^4\right|dx\\
    \le&\int_{x_{3i}}^{x_{3i}+3\Delta x}\frac{(x-x_{3i})^4}{4!}\sup\limits_{[a, b]}\left|f^{(4)}\right|dx\\
    =&\frac{243}{5!}\sup\limits_{[a, b]}\left|f^{(4)}\right|\cdot (\Delta x)^5
\end{flalign*}
thus
\begin{flalign*}
    \;&\left|\int_{x_{3i}}^{x_{3i}+3\Delta x}P_i(x)dx-\int_{x_{3i}}^{x_{3i}+3\Delta x}f(x)dx\right|\\
    \le&\left|\int_{x_{3i}}^{x_{3i}+3\Delta x}f(x)dx-\int_{x_{3i}}^{x_{3i}+3\Delta x}H_i(x)dx\right|+\left|\int_{x_{3i}}^{x_{3i}+3\Delta x}P_i(x)dx-\int_{x_{3i}}^{x_{3i}+3\Delta x}H_i(x)dx\right|\\
    \le& C(\Delta x)^5\sup\limits_{[a, b]}\left|f^{(4)}\right|
\end{flalign*}
Where $C$ is a constant independent of $\Delta x$. This gives the error bound.

\section{}
First we consider $\int_{nT}^{(n+1)T}\left|\frac{f(x)}{x}\right| dx$ where $n\in \N$ and $T>0$.\\
Given $f(x)$ is continuous in $\R$, for any $a\in [nT, (n+1)T]$, we have $\lim\limits_{x\to a}\left|f(x)-f(a)\right|=0$\\
and: $0\le \left| |f(x)|-|f(a)|\right| \le \left|f(x)-f(a)\right|$\\
thus by squeeze theorem we have $\lim\limits_{x\to a}\left| |f(x)|-|f(a)|\right|=0$\\
hence $|f(x)|$ is continuous in $[nT, (n+1)T]$\\
hence $\frac{|f(x)|}{x}$ is continuous in $[nT, (n+1)T]$ since $x>0$ whenever $x\in [nT, (n+1)T]$\\
Moreover we've shown in previous homework that for a continuous non-negative function $g(x)$,
$$\int_{a}^{b}g(x)dx=0 \implies g(x) \equiv 0 $$
hence 
$$\frac{|f(x)|}{x} \not\equiv 0 \implies \int_{nT}^{(n+1)T}\frac{|f(x)|}{x}dx\neq 0$$
thus $\int_{nT}^{(n+1)T}\frac{|f(x)|}{x}dx >0 $.  
Moreover, we know that 
$$\exists c\in [nT, (n+1)T] \st \left|\frac{f(c)}{c}\right|\cdot T = \int_{nT}^{(n+1)T}\frac{|f(x)|}{x}dx $$
thus for such $c$ we have $\left|\frac{f(c)}{c}\right|\cdot T >0$ and 
$$\int_{nT}^{(n+1)T}\frac{|f(x)|}{x}dx=\left|\frac{f(c)}{c}\right|\cdot T > \left|\frac{f(c)}{(n+1)T}\right| \cdot T=\frac{|f(c)|}{n+1}$$
By the fact that $f$ is a periodic function, choose $m, n\in N$, then 
$$\int_{mT}^{(m+1)T}f(x)dx=\int_{nT}^{(n+1)T}f(x)dx$$
thus
$$\int_{mT}^{(m+1)T}\frac{|f(x)|}{x}dx > \frac{|f(c)|}{n+1}$$
This shows for any $n\in \N$ we have $\int_{nT}^{(n+1)T}\frac{|f(x)|}{x}dx > \frac{|f(c)|}{n+1}$\\
Then, choose $m\in \N$ such that $mT>1$ (by \textit{Archimedean Property} such $m$ exists)\\
then for any $n\in \N$ and $n>m$ we have
$$\int_{mT}^{nT}\frac{|f(x)|}{x}dx \le  \int_{1}^{+\oo}\left|\frac{f(x)}{x}\right|dx$$
$$\imply \int_{1}^{+\oo}\left|\frac{f(x)}{x}\right|dx \ge \int_{mT}^{nT}\frac{|f(x)|}{x}dx > \sum_{k=m}^{n-1}\frac{|f(c)|}{k+1}=|f(c)|\cdot \sum_{k=m+1}^{n}\frac{1}{k} $$
Moreover from 1023 we know that $\sum_{k=m}^{+\oo}\frac{1}{k}=+\oo$ diverges, hence $\int_{mT}^{nT}\left|\frac{f(x)}{x}\right| dx=+\oo$, and by comparison test we have
$$\int_{1}^{+\oo}\left|\frac{f(x)}{x}\right|dx=+\oo$$

\section{}
\paragraph{(a)}\ \\
Firstly we have $\frac{t^{s-1}}{e^t-1}>0$, $\frac{t^{s-1}}{e^{nt}}>0$ for $t>0$.\\
For the first one:
\begin{align*}
    \int_{0}^{\infty}\frac{t^{s-1}}{e^t-1}dt&=\int_{0}^{1}\frac{t^{s-1}}{e^t-1}dt+
    \int_{1}^{\infty}\frac{t^{s-1}}{e^t-1}dt\\
    &=\lim_{a\to0} \int_{a}^{1}\frac{t^{s-1}}{e^t-1}dt+
    \lim_{b\to\infty}\int_{1}^{b}\frac{t^{s-1}}{e^t-1}dt
\end{align*}
It suffices to show that both limits converge.\\
By Taylor's expansion:
$$e^t-1>t ,\, e^t-1>\frac{1}{\lceil 2s\rceil!}t^{\lceil 2s\rceil}$$
So 
$$0<\frac{t^{s-1}}{e^t-1}<t^{s-2} , 0<\frac{t^{s-1}}{e^t-1}<{t^{s-1-\lceil 2s\rceil}}\cdot{\lceil 2s\rceil!}$$
For we have:
$$0<\int_{a}^{1}\frac{t^{s-1}}{e^t-1}dt<\int_{0}^{1}
t^{s-2}dt=\left[t^{s-1}/(s-1)\right]^{1}_0(0<a<1)$$
$$0<\int_{1}^{b}\frac{t^{s-1}}{e^t-1}dt<\int_{1}^{\infty}\left({t^{s-1-\lceil 2s\rceil}}\cdot{\lceil 2s\rceil!}\right)dt=\frac{{-\lceil 2s\rceil!}}{(s-\lceil 2s\rceil)}(b>1)$$
$\therefore$ The functions
$$F(x)=\int_{-x}^{1}\frac{t^{s-1}}{e^t-1}dt(-1<x<0),G(x)=\int_{1}^{x}\frac{t^{s-1}}{e^t-1}dt(x>1)$$
are all monotone and bounded\\
$\therefore$ Both limits converge.(i.e.$\int_{0}^{\infty}\frac{t^{s-1}}{e^t-1}dt$ converges)\\
For the second one:\\
$$\int_{0}^{\infty}\frac{t^{s+1}}{e^{nt}}=\int_{0}^{1}\frac{t^{s+1}}{e^{nt}}
+\lim_{b\to\infty}\int_{1}^{b}\frac{t^{s+1}}{e^{nt}}$$
Thus it suffices to prove
$\lim_{b\to\infty}\int_{1}^{b}\frac{t^{s+1}}{e^{nt}}$ converges.\\
$\because \lim_{t\to\infty}\frac{t^{s+1}}{e^{nt}}=0$, $\exists N,$ s.t.$\forall t>N,$
$\frac{t^{s-1}}{t^{s+1}}>\frac{t^{s-1}}{e^{nt}}>0$\\
$\therefore$ By the comparison test, it suffices to prove
$$\int_{1}^{\infty}\frac{t^{s-1}}{t^{s+1}}dt=\int_{1}^{\infty}\frac{1}{t^2}dt$$
converges, which is trivial.

\paragraph{(b)}Proof: \\
\begin{align*}
    n^s\int_{0}^{\infty}t^{s-1}e^{-nt}dt&=
    \lim_{b\to\infty}n^s\int_{0}^{b}t^{s-1}e^{-nt}dt\\
    &=\lim_{b\to\infty}n\int_{0}^{b}(nt)^{s-1}e^{-nt}dt\\
\end{align*}
By substituting $nt$ with $x$ and the composition rule, we have
\begin{align*}
    \lim_{b\to\infty}n\int_{0}^{b}(nt)^{s-1}e^{-nt}dt&=
    \lim_{b\to\infty}\int_{0}^{nb}(x)^{s-1}e^{-x}dx\\
    &=\lim_{nb\to\infty}\int_{0}^{nb}(x)^{s-1}e^{-x}dx\\
    &=\lim_{c\to\infty}\int_{0}^{c}(x)^{s-1}e^{-x}dx\\
    &=\int_{0}^{\infty }t^{s-1}e^{-t}dt\\
    &=\Gamma (s)
\end{align*}
Then by rearrangement:
$$\frac{1}{n^s}\Gamma (s)=\int_{0}^{\infty}t^{s-1}e^{-nt}dt$$



\paragraph{(c)}Proof:\\
\begin{align*}
    \zeta(s)\Gamma(s)&=\sum_{n=1}^{\infty}\frac{1}{n^s}\Gamma(s)\\
\end{align*}
By \textbf{4(b)} we have:
$$\frac{1}{n^s}\Gamma (s)=\int_{0}^{\infty}t^{s-1}e^{-nt}dt$$
Also by the feasibility of swapping the symbols and the formula of geometrical sequence summation, we can get:
\begin{align*}
    \sum_{n=1}^{\infty}\frac{1}{n^s}\Gamma(s)&=\sum_{n=1}^{\infty}\int_{0}^{\infty}t^{s-1}e^{-nt}dt\\
    &=\int_{0}^{\infty}\sum_{n=1}^{\infty}t^{s-1}e^{-nt}dt\\
    &=\int_{0}^{\infty}t^{s-1}(\sum_{n=1}^{\infty}e^{-nt})dt\\
    &=\int_{0}^{\infty}\frac{t^{s-1}}{e^t-1}dt
\end{align*}
To conclude:
\begin{align*}
    \zeta(s)\Gamma(s)&=\int_{0}^{\infty}\frac{t^{s-1}}{e^t-1}dt\\
\end{align*}


\end{document}