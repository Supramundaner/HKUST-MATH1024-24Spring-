\documentclass{article}
\usepackage{amsfonts, amsmath, amsthm, amssymb, color, enumitem, float, fullpage, geometry, graphicx, hyperref, mathrsfs, setspace, subfigure, yhmath}
\title{\LaTeX{} Answer to Problem Set \#}
\author{AO Yuzhuo, JIN Zhibo, LYU Changlai\thanks{names listed in alphabetical order}}
\date{\today}

\newtheorem{Thm}{Theorem}[section]
\newtheorem{Lem}[Thm]{Lemma}
\newtheorem{Prop}[Thm]{Proposition}
\newtheorem{Cor}[Thm]{Corollary}
\newtheorem{Rem}[Thm]{Remark}
\newtheorem{Def}[Thm]{Definition}
\def\sep{\vspace{1cm}\hrule\vspace{1cm}}
\def\tab{\;\;\;\;\;\;}
\def\R{\mathbb{R}}
\def\Q{\mathbb{Q}}
\def\N{\mathbb{N}}
\def\C{\mathbb{C}}
\def\Z{\mathbb{Z}}
\def\a{\alpha}
\def\b{\beta}
\def\c{\gamma}
\def\d{\delta}
\def\e{\epsilon}
\def\p{\partial}
\def\h{\theta}
\def\w{\omega}
\def\bu{\mathbf{u}}
\def\bv{\mathbf{v}}
\def\iff{\Longleftrightarrow}
\def\oif{\Rightarrow}
\def\to{\rightarrow}
\def\inj{\hookrightarrow}
\def\surj{\twoheadrightarrow}
\def\imply{\Longrightarrow}
\def\x{\times}
\def\<{\langle}
\def\>{\rangle}
\def\oo{\infty}
\def\normal{\triangleleft}
\def\h{\hspace*{0.5cm}}
\def\limsupa{\limsup\limits_{n\to\infty}{a_n}}
\def\limsupb{\limsup\limits_{n\to\infty}{b_n}}
\def\liminfa{\liminf\limits_{n\to\infty}{a_n}}
\def\liminfb{\liminf\limits_{n\to\infty}{b_n}}
\def\limsupn#1{\limsup\limits_{n\to\infty}\bigg\left({#1}\bigg\right)}
\def\liminfn#1{\liminf\limits_{n\to\infty}\bigg\left({#1}\bigg\right)}
\def\LIM{\text{LIM}}
\def\seqa{\{a_n\}}
\def\seqb{\{b_n\}}
\def\sseqa{\{a_{n_k}\}}
\def\st{\text{ s.t. } }
\def\when{\text{ when }}
\def\where{\text{ where }}

\begin{document}
\maketitle

\everymath{\displaystyle}

\section{}
\paragraph[short]{(a)}
{By Taylor's expansion, we have 
$$\log\left(1+x\right)=x-\frac{x^2}{2}+\frac{x^3}{3}-\frac{x^4}{4}+\cdots$$
Thus
$$|\log\left(1+x\right)-x|=\left|-\frac{x^2}{2}+\frac{x^3}{3}-\frac{x^4}{4}+\cdots\right|=\left|\sum_{i=2}^{\infty} \frac{\left(-1\right)^{i-1}x^i}{i}\right|$$
Since $\forall i\in\mathbb{N}$, $ \left|\frac{\left(-1\right)^{i-1}x^i}{i}\right|<|x|^i$ and $x\in\left(-1,1\right)$, 
we have
$$|\log\left(1+x\right)-x| \leq \sum_{i=2}^{\infty}
    \left|\frac{\left(-1\right)^{i-1}x^i}{i}\right| \leq
    \sum_{i=2}^{\infty}|x|^i=\frac{{|x|}^2}{1-|x|}
    $$
Thus we finish the first inequality
$$
\log\left(1+x\right)-x\leq\frac{{|x|}^2}{1-|x|}$$
For the second equality:
By Taylor's expansion and Lagrange remainder, we have
$$\log\left(1+x\right)=x-\frac{x^2}{2}+\frac{x^3}{3}-\frac{x^4}{4}+\cdots+\frac{\left(-1\right)^{n-1}x^n}{n}+R_n\left(x\right)$$
where $R_n\left(x\right)=\frac{\left(-1\right)^{n-1}x^{n+1}}{\left(n+1\right)\left(1+c\right)^{n+1}}$ for some $c\in\left(-1,1\right)$
and 
$$\log\left(1-x\right)=
-x-\frac{x^2}{2}-\frac{x^3}{3}-\frac{x^4}{4}-\cdots-\frac{x^n}{n}+R_n\left(-x\right)$$
where $R_n\left(-x\right)=
\frac{\left(-1\right)^{n}\left(-x\right)^{n+1}}{\left(n+1\right)\left(1+c\right)^{n+1}}$ for some $c\in\left(-1,1\right)$\\
Thus $$R_2\left(x\right)=\frac{x^3}{3\left(1+c\right)^3} \in O\left(x^3\right)$$ 
$$R_2\left(-x\right)=-\frac{x^3}{3\left(1+c\right)^3} \in O\left(x^3\right)$$
Thus 
$$\log\left(1+x\right)-\log\left(1-x\right)=2x+O\left(x^3\right)$$
Namely
$$\log\frac{1+x}{1-x}=2x+O\left(x^3\right)$$
Thus we finish the second equality.

}

\paragraph[short]{(b)}
{
    $$\begin{aligned}
        \forall n \in \mathbb{N} ,\, \text{RHS} = &\int_{\frac{1}{2}}^{n+\frac{1}{2}} \log t \, dt + \sum_{k=1}^{n}\left(\log k-\int_{k-\frac{1}{2}}^{k+\frac{1}{2}} \log t \, dt\right) \\
        = &\int_{\frac{1}{2}}^{n+\frac{1}{2}} \log t \, dt + \sum_{k=1}^{n} \log k - \sum_{k=1}^{n} \int_{k-\frac{1}{2}}^{k+\frac{1}{2}} \log t \, dt \\
        = &\sum_{k=1}^{n} \log k \\
        = &\log \left(n!\right) = \text{LHS}
    \end{aligned}$$
}

\paragraph[short]{(c)}
{
Firstly we have:
$$  \int_{}^{}\log t \, dt=t\log t-t+C$$
Thus $$\int_{\frac{1}{2}}^{n+\frac{1}{2}}\log t \, dt=
\left(n+\frac{1}{2}\right)\log\left(n+\frac{1}{2}\right)+\frac{1}{2}\log2-n$$
Thus
$$\int_{\frac{1}{2}}^{n+\frac{1}{2}}\log t \, dt-\left(\left(n+\frac{1}{2}\right)\log n
-n+\frac{1+\log 2}{2}+O\left(\frac{1}{n}\right)\right)=\left(n+\frac{1}{2}\right)\log \left(1+\frac{1}{2n}\right)-\frac{1}{2}$$
Then by Taylor's expansion and Lagrange remainder, we have
\begin{align*}
    \left(n+\frac{1}{2}\right)\log \left(1+\frac{1}{2n}\right)-\frac{1}{2}&=\left(n+\frac{1}{2}\right)\left(\frac{1}{2n}+O\left(\frac{1}{4n^2}\right)\right)-\frac{1}{2}\\
    &=\frac{1}{2}+O\left(\frac{1}{n}\right)+\frac{1}{4n}+O\left(\frac{1}{n^2}\right)-\frac{1}{2}\\
    &=O\left(\frac{1}{n}\right)
\end{align*}
Thus we have
$$\int_{\frac{1}{2}}^{n+\frac{1}{2}}\log t \, dt=
\left(n+\frac{1}{2}\right)\log\left(n\right)+\frac{1+\log 2}{2}-n+O\left(\frac{1}{n}\right)$$
}

\paragraph[short]{(d)}
{
  Firstly, by 
  $$\int \log t \, dt=t\log t-t+C$$
  we have
  $$\int_{k}^{k+\frac{1}{2}}\log t \, dt =
    \left(k+\frac{1}{2}\right)\log\left(k+\frac{1}{2}\right)-k\log k-\frac{1}{2}$$
and
$$\int_{k-\frac{1}{2}}^{k}\log t \, dt =
    k\log k-\left(k-\frac{1}{2}\right)\log\left(k-\frac{1}{2}\right)-\frac{1}{2}$$
Thus   
\begin{align*}
    \left|\frac{1}{2}\log k-\int_{k}^{k+\frac{1}{2}}\log t \, dt\right|&=\left|\frac{1}{2}\log k-\left(k+\frac{1}{2}\right)\log\left(k+\frac{1}{2}\right)+k\log k+\frac{1}{2}\right|\\
    &=\left(k+\frac{1}{2}\right)\log\left(1+\frac{1}{2k}\right)-\frac{1}{2}\\
    &=\left(k+\frac{1}{2}\right)\left(\frac{1}{2k}+\frac{1}{8k^2}+O\left(\frac{1}{k^3}\right)\right)-\frac{1}{2}\\
    &=\frac{1}{8k}+O\left(\frac{1}{k^2}\right)
\end{align*}
Similarly, we have
\begin{align*}
    \left|\frac{1}{2}\log k-\int_{k-\frac{1}{2}}^{k}\log t \, dt\right|&=\left|\frac{1}{2}\log k-k\log k+\left(k-\frac{1}{2}\right)\log\left(k-\frac{1}{2}\right)+\frac{1}{2}\right|\\
    &=\left| \left(k-\frac{1}{2}\right)\log\left(1-\frac{1}{2k}\right) + \frac{1}{2} \right| \\
    &=\left(k-\frac{1}{2}\right)\left(-\frac{1}{2k}+\frac{1}{8k^2}+O\left(\frac{1}{k^3}\right)\right)+\frac{1}{2}\\
    &=\frac{1}{8k}+O\left(\frac{1}{k^2}\right)
\end{align*}
Then combining the above two equations, we have
\begin{align*}
    \log k-\int_{k-\frac{1}{2}}^{k+\frac{1}{2}}\log t \, dt&= \frac{1}{2}\log k-\int_{k}^{k+\frac{1}{2}}\log t \, dt+\frac{1}{2}\log k-\int_{k-\frac{1}{2}}^{k}\log t \, dt\\
    &=-\left|\frac{1}{2}\log k-\int_{k}^{k+\frac{1}{2}}\log t \, dt\right|+\left|\frac{1}{2}\log k-\int_{k-\frac{1}{2}}^{k}\log t \, dt\right|\\
    &=-\left(\frac{1}{8k}+O\left(\frac{1}{k^2}\right)\right) + \left(\frac{1}{8k}+O\left(\frac{1}{k^2}\right)\right)\\
    &=O\left(\frac{1}{k^2}\right)
\end{align*}
Therefore,
$$\exists C\in\mathbb{R^+} \text{ s.t. } \forall k\in\mathbb{N} ,\, \left|\log k-\int_{k-\frac{1}{2}}^{k+\frac{1}{2}}\log t \, dt \right| < \frac{C}{k^2}$$
Then by the comparison test, we have $\sum_{k=1}^{\infty}\left|\log k-\int_{k-\frac{1}{2}}^{k+\frac{1}{2}}\log t \, dt \right|$ converges.
\\
Thus by the absulute convergence test, we have the convergence of the series:
$$\sum_{k=1}^{\infty}\left(\log k-\int_{k-\frac{1}{2}}^{k+\frac{1}{2}}\log t \, dt \right)$$
}

\paragraph[short]{(e)}
{
By the above analysis, we have
$$\int_{\frac{1}{2}}^{n+\frac{1}{2}}\log t \, dt=
\left(n+\frac{1}{2}\right)\log\left(n\right)+\frac{1+\log 2}{2}-n+O\left(\frac{1}{n}\right)$$
\begin{align*}
    \log n!&=\int_{\frac{1}{2}}^{n+\frac{1}{2}}\log t \, dt+\sum_{k=1}^{n}\left(\log k-\int_{k-\frac{1}{2}}^{k+\frac{1}{2}}\log t \, dt\right)\\
    &=\int_{\frac{1}{2}}^{n+\frac{1}{2}}\log t \, dt+\sum_{k=1}^{n}\left(\log k-\int_{k-\frac{1}{2}}^{k+\frac{1}{2}}\log t \, dt\right)\\
    &=\left(n+\frac{1}{2}\right)\log n+\frac{1+\log 2}{2}-n+O\left(\frac{1}{n}\right)+\sum_{k=1}^{n}\left(\log k-\int_{k-\frac{1}{2}}^{k+\frac{1}{2}}\log t \, dt\right)\\
\end{align*}
Thus
\begin{align*}
    \log n!-\left(n+\frac{1}{2}\right)\log n+n&=\frac{1+\log 2}{2}+O\left(\frac{1}{n}\right)+\sum_{k=1}^{n}\left(\log k-\int_{k-\frac{1}{2}}^{k+\frac{1}{2}}\log t \, dt\right)\\
    &=\frac{1+\log 2}{2}+\sum_{k=1}^{n}\left(\log k-\int_{k-\frac{1}{2}}^{k+\frac{1}{2}}\log t \, dt\right)+O\left(\frac{1}{n}\right)\\
\end{align*}
Thus by the convergence of 
$$\sum_{k=1}^{\infty}\left(\log k-\int_{k-\frac{1}{2}}^{k+\frac{1}{2}}\log t \, dt \right),$$
$\left(\log n!-\left(n+\frac{1}{2}\right)\log n+n\right)$
converges to a finite value.
Thus
$$\lim_{n\to\infty}\log \left(\frac{n!e^n}{n^{n+\frac{1}{2}}}\right)$$
exists.
By the continuity of the exponential function, we have
$$\lim_{n\to\infty}\frac{n!e^n}{n^{n+\frac{1}{2}}}=:L$$
exists.
}

\paragraph[short]{(f)}
{
We can prove them by induction.\\
For $n=0$, we have
$$I_0=\int_{0}^{\frac{\pi}{2}}1 dt=\frac{\pi}{2}=C^0_0\frac{\pi}{2^1}$$
$$I_1=\int_{0}^{\frac{\pi}{2}}\sin t dt=1=\frac{2^0\left(0!\right)^2}{\left(1\right)!}$$
Then if we have:
$$I_{2n}=C^{2n}_n\frac{\pi}{2^{2n+1}}$$
$$I_{2n+1}=\frac{2^{2n}\left(n!\right)^2}{\left(2n+1\right)!}$$
Then:
\begin{align*}
    I_{2n+2}&=\int_{0}^{\frac{\pi}{2}}\sin^{2n+2}t dt\\
    &=\int_{0}^{\frac{\pi}{2}}\sin^{2n+1}t\sin t dt\\
    &=\int_{0}^{\frac{\pi}{2}}\sin^{2n+1}t d\left(-\cos t\right)\\
    &=\left.\sin^{2n+1}t\left(-\cos t\right)\right|_{0}^{\frac{\pi}{2}}-\int_{0}^{\frac{\pi}{2}}\left(-\cos t\right)d\sin^{2n+1}t\\
    &=\left(2n+1\right)\int_{0}^{\frac{\pi}{2}}\sin^{2n}t\cos^2t dt\\
    &=\left(2n+1\right)\int_{0}^{\frac{\pi}{2}}\sin^{2n}t\left(1-\sin^2t\right) dt\\
    &=\left(2n+1\right)\int_{0}^{\frac{\pi}{2}}\sin^{2n}t dt-\left(2n+1\right)\int_{0}^{\frac{\pi}{2}}\sin^{2n+2}t dt\\
    &=\left(2n+1\right)I_{2n}-\left(2n+1\right)I_{2n+2}
\end{align*}
Thus
$$I_{2n+2}=\frac{2n+1}{2n+2}I_{2n}=C^{2\left(n+1\right)}_{n+1}\frac{\pi}{2^{2\left(n+1\right)+1}}$$
\begin{align*}
    I_{2n+3}&=\int_{0}^{\frac{\pi}{2}}\sin^{2n+3}t dt\\
    &=\int_{0}^{\frac{\pi}{2}}\sin^{2n+2}t\sin t dt\\
    &=\int_{0}^{\frac{\pi}{2}}\sin^{2n+2}t d\left(-\cos t\right)\\
    &=\left.\sin^{2n+2}t\left(-\cos t\right)\right|_{0}^{\frac{\pi}{2}}-\int_{0}^{\frac{\pi}{2}}\left(-\cos t\right)d\sin^{2n+2}t\\
    &=\left(2n+2\right)\int_{0}^{\frac{\pi}{2}}\sin^{2n+1}t\cos^2t dt\\
    &=\left(2n+2\right)\int_{0}^{\frac{\pi}{2}}\sin^{2n+1}t\left(1-\sin^2t\right) dt\\
    &=\left(2n+2\right)\int_{0}^{\frac{\pi}{2}}\sin^{2n+1}t dt-\left(2n+2\right)\int_{0}^{\frac{\pi}{2}}\sin^{2n+3}t dt\\
    &=\left(2n+2\right)I_{2n+1}-\left(2n+2\right)I_{2n+3}
\end{align*}
Thus
$$I_{2n+3}=\frac{2n+2}{2n+3}I_{2n+1}=\frac{2^{2\left(n+1\right)}\left(n+1\right)!^2}{\left(2n+3\right)!}$$
Thus by induction, we have
$$I_{2n}=
C^{2n}_n\frac{\pi}{2^{2n+1}}$$
$$I_{2n+1}=\frac{2^{2n}\left(n!\right)^2}{\left(2n+1\right)!}$$
}

\paragraph[short]{(g)}
{
Firstly, we need to prove that
$$\lim_{n\to\infty}\frac{I_{2n}}{I_{2n+1}}=1,$$
By $0<\sin^{n+1} t\leq \sin^{n}t$ for $t\in(0,\frac{\pi}{2})$, we have
$$0<I_{n+1}\leq I_{n}$$
Thus the sequence $\{I_n\}$ is monotonic decreasing.
So we have:
$$0<I_{2n-1}\leq I_{2n}\leq I_{2n+1}\leq I_{2n+2}$$
Thus
$$\frac{I_{2n-1}}{I_{2n+1}}\leq \frac{I_{2n}}{I_{2n+1}}\leq \frac{I_{2n}}{I_{2n+2}}$$
By 1.(f)
$$I_{2n+2}=\frac{2n+1}{2n+2}I_{2n}=C^{2\left(n+1\right)}_{n+1}\frac{\pi}{2^{2\left(n+1\right)+1}}$$
$$I_{2n+3}=\frac{2n+2}{2n+3}I_{2n+1}=\frac{2^{2\left(n+1\right)}\left(n+1\right)!^2}{\left(2n+3\right)!}$$
So
$$\lim_{n\to\infty}\frac{I_{2n+1}}{I_{2n+3}}=\lim_{n\to\infty}\frac{I_{2n-1}}{I_{2n+1}}=\lim_{n\to\infty}\frac{I_{2n}}{I_{2n+2}}=1$$
Then by the squeeze theorem, we have
$$\lim_{n\to\infty}\frac{I_{2n}}{I_{2n+1}}=1$$
Thus 
$$\lim_{n\to\infty}\frac{C^{2n}_n\frac{\pi}{2^{2n+1}}}{\frac{2^{2n}\left(n!\right)^2}{\left(2n+1\right)!}}=1$$
In other words,
$$\lim_{n\to\infty}\frac{\left(n!\right)^42^{4n}}{\left(2n\right)!\left(2n+1\right)!}=\frac{\pi}{2}$$
Also by the definition of $L$:
$$L=\sqrt{2\pi}\iff \lim_{n\to\infty}\frac{\left(n!\right)^2e^{2n}}{4\cdot n^{2n+1}}=\frac{\pi}{2}$$
Thus
\begin{align*}
L=\sqrt{2\pi}&\iff \lim_{n\to\infty}\frac{\left(n!\right)^2e^{2n}}{4\cdot n^{2n+1}}=\lim_{n\to\infty}\frac{\left(n!\right)^42^{4n}}{\left(2n\right)!\left(2n+1\right)!}\\
&\iff \lim_{n\to\infty}\frac{4\left(n!\right)^2\left(\left(\frac{16}{e^2}\right)^n\cdot n^{2n+1}\right)}{\left(2n\right)!\left(2n+1\right)!}=1\\
&\iff \lim_{n\to\infty}\frac{
    4\left(\frac{L\cdot n^{n+\frac{1}{2}}}{e^n}\right)^2\left(\frac{16}{e^2}\right)^n\cdot n^{2n+1}
}{
    \frac{L\cdot\left(2n\right)^{2n+\frac{1}{2}}}{e^{2n}}\cdot\frac{L\cdot\left(2n\right)^{2n+\frac{1}{2}}}{e^{2n}}\cdot\left(2n+1\right)
}=1\\
&\iff \lim_{n\to\infty}\frac{
    4\cdot n^{4n+2}\left(16\right)^n
}{
    2^{4n+1}\cdot n^{4n+1}\cdot\left(2n+1\right)
}=1\\
&\iff \lim_{n\to\infty}\frac{2n}{2n+1}=1
\end{align*}
It's trivial to prove that
$$\lim_{n\to\infty}\frac{2n}{2n+1}=1$$
Thus we finish the proof.
}

\section{}
Claim: above series converges if and only if $|x|<\frac{27}{4} $. \\
\textbf{Justification:}
\paragraph{Case 1} $x<\frac{27}{4}$: \\
if $x=0$, then the series converges to 0.\\
when $0<x<\frac{27 }{4} $, let $a_n$ be the $n$-th term of the series, and consider the ratio of terms:
$$\frac{a_{n+1}}{a_n}=x\cdot \frac{\frac{(n+1)!(2n+2)!}{(3n+3)!} }{\frac{n!(2n)!}{(3n)!} }=x\cdot \frac{(n+1)(2n+1)(2n+2)}{(3n+3)(3n+2)(3n+1)}=x\cdot \frac{4n^2+6n+2}{27n^2+27n+6}$$
thus given $0<x<\frac{27}{4}$:
$$\lim\limits_{ n\to +\oo } x\cdot \frac{4n^2+6n+2}{27n^2+27n+6}=x\cdot \frac{4}{27} < 1$$
hence by ratio test, the series converges.

\paragraph{Case 2} $-\frac{27}{4}<x<0$: \\
In this case we have $|x|<\frac{27}{4} $, and we've shown is case 1 (using the ratio test) that series
$$\sum_{n=1}^{+\oo}\frac{|x|^n(2n)!n!}{(3n)!} $$
converges for $|x|<\frac{27}{4} $, thus by absolute convergence, the original series converges.

\paragraph{Case 3} $x\ge \frac{27}{4}$: \\
first we consider each term respectively.\\
Using Stirling's approximation, we see that 
$$\lim\limits_{n\to +\oo}\frac{\sqrt{6\pi n} (\frac{3n}{e} )^{n} }{(3n)!}=\lim\limits_{n\to +\oo}\frac{\sqrt{4\pi n}(\frac{2n}{e} )^{2n}}{(2n)!}=\lim\limits_{n\to +\oo} \frac{\sqrt{2\pi n}(\frac{n}{e})^n }{n!}=1$$
then consider the $n-th$ term $a_n$:\\
$$a_n=\frac{x^n(2n)!n!}{(3n)!} \ge (\frac{27}{4})^n \frac{(2n)!n!}{(3n)!}=:\frac{1}{b_n} $$
and 
\begin{flalign*}
    \;&\lim\limits_{n\to +\oo} b_n\\  
    =&\lim\limits_{n\to +\oo } \frac{(3n)!}{(2n)!n!}\cdot \left(\frac{4}{27}\right)^n\\
    =&\lim\limits_{n\to +\oo } \left(\frac{4}{27}\right)^n\cdot \frac{\sqrt{ 6\pi n }\cdot (\frac{3n}{e})^n}{\sqrt{2\pi n}\cdot (\frac{n}{e} )^n\cdot \sqrt{4\pi n }\cdot (\frac{ 2n }{e})^{2n} }\\ 
    =&\lim\limits_{n\to +\oo}\frac{\sqrt{3}}{\sqrt{4\pi n}}\\
    =&0  
\end{flalign*}
hence, given $b_n>0$ for each $n$, $$\lim\limits_{n\to +\oo } \frac{1}{b_n}=+\oo$$
thus by comparison rules and the fact that $a_n\ge\frac{1}{b_n} >0$ for each $n$, we have 
$$\lim\limits_{n\to +\oo } a_n=+\oo$$
This shows each term of the original series diverges to $+\oo$, hence the series diverges.
\paragraph{case 4} $x\le -\frac{27}{4}$: \\
In this case we have $|x|>\frac{27}{4} $, and we've shown is case 3 that $\frac{|x|^n (2n)!(n)!}{3n!}$ diverges to $+\oo$, hence in this case, the $n-th$ term of the original series diverges to $+\oo$ or $-\oo$ as n goes to $+\oo$,
thus the series diverges.\\\\
In conclusion, the series converges if and only if $|x|<\frac{27}{4} $.   

\section{}
Given $\sum_{k=1}^{n}\phi(k)=\frac{3 }{\pi^2 } n^2+o(n^2)$, write $\sum_{k=1}^{n}\phi(k)$ as $\frac{ 3}{\pi^2} n^2+b_n$, then 
$$b_n=\sum_{k=1}^{n}\phi(k)-\frac{3 }{\pi^2} n^2\in o(n^2)$$
Next we computing $\sum_{k=1}^{n}\frac{\phi(k)}{k^2} $ using summation by parts:
\begin{flalign*}
    \;&\sum_{k=1}^{n}\frac{\phi(k)}{k^2}\\
    =&\sum_{k=1}^{n}\phi(k)\cdot \frac{1 }{ n^2 } -\sum_{k=1}^{n-1}\left(\sum_{j=1}^{k}\phi(j)\right)\cdot \left(\frac{1}{(k+1)^2}-\frac{1}{k^2}\right)\\
    =&\frac{b_n+\frac{3}{\pi^2} n^2}{n^2} +\sum_{k=1}^{n-1}\left(b_k+\frac{3}{\pi^2} k^2\right)\cdot \left(\frac{1}{k^2}-\frac{1}{(k+1)^2}\right)\\
    =&\frac{3 }{\pi^2} +\frac{b_n}{n^2} +\sum_{k=1}^{n-1}\left(\frac{3 }{\pi^2 } k^2\right)\cdot \left(\frac{2k+1 }{k^2(k+1)^2} \right)+\sum_{k=1}^{n-1}b_k\cdot \left(\frac{2k+1}{k^2(k+1)^2}\right)\\
    =&\frac{ 3 }{\pi^2 } +\frac{b_n }{n^2} +\frac{ 3 }{\pi^2} \cdot \sum_{k=1}^{n-1}\left( \frac{2 }{k+1} -\frac{1 }{(k+1)^2} \right)+\sum_{k=1}^{n-1}b_k\cdot \left(\frac{2k+1}{k^2(k+1)^2}\right)\\
    =&\frac{ 3 }{\pi ^2} +\frac{b_n}{n^2} +\frac{ 6 }{\pi^2} \sum_{k=1}^{n-1}\frac{1 }{k } -\frac{3 }{\pi^2 } - \frac{ 3 }{\pi ^2 }\cdot \sum_{k=1}^{n-1}\frac{1 }{(k+1)^2} +\sum_{k=1}^{n-1}b_k\cdot \left(\frac{2k+1}{k^2(k+1)^2}\right)\\
    =&\frac{b_n}{n^2} +\frac{ 6 }{\pi^2} \sum_{k=1}^{n}\frac{1 }{k } -\frac{ 3 }{\pi ^2 }\cdot \sum_{k=1}^{n-1}\frac{1 }{(k+1)^2} +\sum_{k=1}^{n-1}b_k\cdot \left(\frac{2k+1}{k^2(k+1)^2}\right)\\
\end{flalign*}
then we consider the above summation term by term:\\
For the first term $\frac{b_n}{n^2}$, since $b_n\in o(n^2)$, we have $\frac{b^n}{n^2}\to 0$ as $n\to +\oo$, and hence $\frac{b_n}{n^2}\in o(1)$\\
for the second term $\frac{ 6 }{\pi^2} \sum_{k=1}^{n}\frac{1 }{k }$, by the fact that 
$$\sum_{k=1}^{n}\frac{1 }{k }=\log(n)+\gamma+o(1)$$
we see that 
$$\frac{6 }{ \pi^2 }\sum_{k=1}^{n}\frac{1 }{k } = \frac{6 }{\pi ^2 } \log n +\frac{6 }{\pi^2 }\gamma+o(1)$$
and we know that $\log n \to +\oo $ as $n\to +\oo$, hence 
$$\frac{ 6 }{\pi^2 } \sum_{k=1}^{n}\frac{1}{k} = \frac{ 6}{ \pi^2 } \log n+ o(\log n)$$
for the third term $-\frac{ 3 }{\pi^2 }\sum_{k=1}^{n-1}\frac{1 }{(k+1)^2}$: we know that 
$$\sum_{k=1}^{n}\frac{ 1 }{k^2 } = \frac{\pi ^2 }{6 } + o(1)$$
hence 
$$-\frac{3}{\pi^2}\sum_{k=1}^{n-1}\frac{1}{(k+1)^2} = -\frac{3}{\pi^2 }\left(\sum_{k=1}^{n}\frac{1 }{k^2 } -1\right) = -\frac{1}{2}+\frac{\pi^2}{3}+o(1)=o(\log n)$$
for the last term $\sum_{k=1}^{n-1}b_k\cdot \left(\frac{2k+1}{k^2(k+1)^2}\right)$:\\
let $c_k:= \frac{|b_k|}{k^2} \cdot \frac{ 2  }{ k }$, then
$$\left|b_k\cdot\frac{2k+1}{k^2(k+1)^2}\right| =\left|\frac{b_k}{k^2}\right|\cdot\left|\frac{2}{(k+1)}-\frac{1}{(k+1)^2}\right| < \frac{|b_k|}{k^2} \cdot \frac{2}{ k} = c_k$$ 
Since $b_k\in o(k^2)$, for $c_k$ we have
$$\lim\limits_{n\to +\oo } \frac{c_k }{\frac{1 }{k} } = \lim\limits_{n\to +\oo }\frac{2|b_k|}{k^2} =0$$
Next we compare the series $\sum_{k=1}^{n}c_k$ with $\sum_{k=1}^{n}\frac{1 }{k}$.\\
Choose $\e>0$.\\\\
Given $\dfrac{c_k }{\frac{1 }{k} } \to 0$, $\exists N_1\in \N \st \forall n>N_1,~ \dfrac{c_n}{\frac{ 1 }{n } } <\frac{\e }{4 } $\\
Moveover, $\sum_{n=1}^{+\oo} \frac{ 1 }{n}$ diverges to $+\oo$, hence $\exists N_2 \st \forall n>N_2,~ \frac{\sum_{k=1}^{N_1}c_k}{\sum_{k=1}^{n}\frac{1 }{k } }<\frac{\e}{4} $\\
Let $N=\max\{N_1, N_2\}$, then $\forall n>N$, 
$$\frac{\sum_{k=1}^{n}c_k  }{\sum_{k=1}^{n}\frac{ 1 }{k } }= \frac{\sum_{k=1}^{N_1}c_k }{\sum_{k=1}^{n }\frac{ 1 }{k}}+\frac{ \sum_{k=N_1+1 }^{n}c_k}{\sum_{k=1}^{n}\frac{ 1 }{k} }<\frac{ \e }{4} +\frac{\e \cdot (n-N_1 )}{4n}<\e $$
hence for $\e$, we've found $N$ such that $\frac{\sum_{k=1}^{n}c_k }{\sum_{k=1}^{n}\frac{ 1}{k } }<\e$ holds for all $n>N$, and since $\e$ can be arbitrarily small, we've shown that 
$$\lim\limits_{ n\to +\oo } \frac{ \sum_{k=1}^{n }c_k }{\sum_{k=1}^{n}\frac{ 1 }{k } }=0$$
and we see that 
$$\lim\limits_{ n\to +\oo } \frac{ \sum_{k=1}^{n}\frac{1 }{k}  }{\log n}=1$$
hence 
$$\lim\limits_{ n\to +\oo } \frac{\sum_{k=1}^{n }\frac{ 1 }{k } c_k }{\sum_{k=1}^{n }\frac{ 1 }{k } }\cdot \frac{\sum_{k=1}^{n }\frac{ 1 }{k} }{\log n}=0\cdot 1=0$$
this shows
$$\lim\limits_{ n\to +\oo } \frac{ \sum_{k=1}^{n }c_k}{\log n}=0$$
moreover, for each $n\ge 3$:
$$0<\left|b_k \cdot \frac{ 2k+1 }{k^2(k+1)^2} \right|<|c_k| \imply 0<\frac{ 1 }{\log n}  \sum_{k=1}^{n}\left|b_k\cdot \frac{ 2k+1}{k^2(k+1)^2} \right|<\frac{ 1 }{\log n} \sum_{k=1}^{n}|c_k|$$
hence by squeeze theorem,
$$\lim\limits_{n\to +\oo } \frac{\sum_{k=1}^{+\oo}\left|b_k\cdot \frac{2k+1}{k^2(k+1)^2}\right|}{\log n}=0\imply \lim\limits_{n\to +\oo } \frac{\sum_{k=1}^{+\oo}b_k\cdot \frac{2k+1}{k^2(k+1)^2}}{\log n}=0$$
this shows $\sum_{k=1}^{n-1}b_k\cdot \left(\frac{2k+1}{k^2(k+1)^2}\right)\in o(\log n)$\\
hence
\begin{flalign*}
    \;&\sum_{k=1}^{n}\frac{\phi(k)}{k^2}\\
    =&\frac{b_n}{n^2} +\frac{ 6 }{\pi^2} \sum_{k=1}^{n}\frac{1 }{k } -\frac{ 3 }{\pi^2 }\sum_{k=1}^{n-1}\frac{1 }{(k+1)^2} +\sum_{k=1}^{n-1}b_k\cdot \left(\frac{2k+1}{k^2(k+1)^2}\right)\\
    =&o(1)+\frac{ 6 }{\pi^2} \log n +o(\log n)+o(\log n)\\
    =&\frac{ 6 }{\pi^2} \log n +o(\log n)
\end{flalign*}
hence, let $C:=\frac{ \pi^2 }{6}$, then 
$$\sum_{k=1}^{n}\frac{\phi(k)}{k^2}=C\log n +o(\log n)$$\\
We'll then show that $\sum_{k=1}^{n}\frac{\phi(k)}{k^s} $ converges when $s>2$.\\
Let $d_n:=\sum_{k=1}^{n}\frac{\phi(k)}{k^2}$.\\
Using summation by parts we see that:
\begin{flalign*}
    \;&\sum_{k=1}^{n}\frac{ \phi(k)}{k^s}\\
    =&\sum_{k=1}^{n}\frac{\phi(k)}{k^2 } \cdot \frac{1 }{k^{s-2}} \\
    =&\sum_{k=1}^{n}\frac{\phi(k)}{k^2} \cdot \frac{ 1}{n^{s-2}} -\sum_{k=1}^{n-1}\left(\sum_{j=1}^{k}\frac{\phi(j)}{j^2} \right)\cdot \left(\frac{ 1}{(k+1)^{s-2}}-\frac{1}{k^{s-2}} \right)\\
    =&d_n\cdot \frac{1}{n^{s-2}}+\sum_{k=1}^{n-1}d_k\cdot \left(\frac{1 }{k^{s-2}} -\frac{ 1}{(k+1)^{s-2 }}\right)
\end{flalign*}
We've shown that $d_n=C\log n +o(\log n)$, hence $\lim\limits_{n\to +\oo } \frac{d_n}{\log n}=C$, and given $s>2$, $s-2>0$, thus 
$$\lim\limits_{n\to +\oo } \frac{\log n}{n^{s-2}}=0\imply\lim\limits_{n\to +\oo } \frac{d_n}{n^{s-2}}=0$$
This shows the convergence of the first term.\\
For the second term $\sum_{k=1}^{n-1}d_k\cdot (\frac{1 }{k^{s-2}} -\frac{1 }{(k+1)^{s-2}})$:\\
let $h$ be a function defined on $[1,+\oo)$ that $h(x):=\frac{1}{x^{s-2}}$, then 
$$h(k)-h(k+1)=h'(y)\cdot (-1)=(s-2)\cdot \frac{1}{y^{s-1}}$$
where $y\in (k, k+1)$, hence
$$0<\frac{1}{k^{s-2}}-\frac{ 1}{(k+1)^{s-2}} = (s-2)\cdot \frac{1}{y^{s-1}}<\frac{s-2}{k^{s-1}}$$
Since $s>2$, $\frac{1}{2} (s-2)>0$,
$$\lim\limits_{n\to +\oo } \frac{\log k}{k^{\frac{1}{2} (s-2)}}=0$$
hence we can find $N_1$ such that 
$$\forall k>N_1,~ \frac{\log k}{k^{\frac{1}{2} (s-2)}}<1$$
and since $\lim\limits_{k\to +\oo } \frac{d_k}{\log k}=C$, we can find $N_2$ such that
$$\forall k>N_2,~ \frac{d_k}{\log k}<2C$$
let $N=\max\{N_1, N_2\}$, then $\forall k>N$:
\begin{flalign*}
    \;&\left|d_k\cdot\left(\frac{1}{k^{s-2}} -\frac{1}{(k+1)^{s-2}} \right)\right|\\
    <&|d_k|\cdot \left(\frac{s-2}{k^{s-1}} \right)\\
    <&\frac{s-2}{k^{s-1}} \cdot (2C)\cdot \log k\\
    <&\frac{s-2}{k^{s-1}}\cdot (2C)\cdot k^{\frac{1}{2} (s-2)}\\
    =&\frac{1}{k^{\frac{s}{2}}}\cdot (2C)\cdot (s-2)\\
\end{flalign*}
Moreover, since $\frac{s}{2}>1$, we see that the series $\sum_{n=1}^{+\oo}\frac{1}{n^{s/2}}$ converges (justified by integral test in 1024 or MVT in 1023) hence $\sum_{k=1}^{n}2C(s-2)\cdot \frac{1}{k^{s-2}} $ converges.\\
Thus, by comparison test, $\sum_{k=1}^{n-1}\left|d_k\cdot (\frac{1}{k^{s-2}} -\frac{1}{(k+1)^{s-2}})\right|$ converges, and by absolute convergence, $\sum_{k=1}^{n-1}d_k\cdot (\frac{1}{k^{s-2}} -\frac{1}{(k+1)^{s-2}})$ converges as well.\\
This shows that $\sum_{k=1}^{n}\frac{\phi(k)}{k^s}=d_n\cdot \frac{1}{n^{s-2}}+\sum_{k=1}^{n-1}d_k\cdot (\frac{1}{k^{s-2}} -\frac{1}{(k+1)^{s-2}})$ converges when $s>2$.\\

\section{}  % Problem 4
\paragraph{(a)}
Denote $a_N := p_1 + p_2 + \cdots + p_N$ and $b_N := q_1 + q_2 + \cdots + q_N$.
\\
Recall Exercise 5.8: $x_n := \sum_{k=1}^{n} f(k) - \int_{1}^{n} f(t) \mathrm{dt}$ converges.
Take $f(t) = \frac{1}{2t-1}$:
$$\begin{aligned}
    x_{a_N} &= 1 + \frac{1}{3} + \cdots + \frac{1}{2a_N-1} - \int_{1}^{a_N} \frac{1}{2t-1} \mathrm{dt} \\
    &= 1 + \frac{1}{3} + \cdots + \frac{1}{2a_N-1} - \frac{1}{2}\ln(2a_N-1) \to \gamma_1 \in \mathbb{R}
\end{aligned}$$
Take $f(t) = \frac{1}{2t}$: $x_{b_N} = \frac{1}{2} + \frac{1}{4} + \cdots + \frac{1}{2b_N} - \frac{1}{2}\ln(2b_N) \to \gamma_2 \in \mathbb{R}$
\\
Therefore:
$$\begin{aligned}
    &s_N - \frac{1}{2} \ln{\frac{p_1 + p_2 + \cdots + p_N}{q_1 + q_2 + \cdots + q_N}} \\
    =& 1 + \frac{1}{3} + \cdots + \frac{1}{2a_N-1} - \left( \frac{1}{2} + \frac{1}{4} + \cdots + \frac{1}{2b_N} \right) - \frac{1}{2} \ln{\frac{a_N}{b_N}} \\
    =& \frac{1}{2} \ln(2a_N-1) + \gamma_1 + o(1) - \left( \frac{1}{2}\ln(2b_N) + \gamma_2 + o(1) \right) - \frac{1}{2} \ln{\frac{a_N}{b_N}} \\
    =& \frac{1}{2} \ln\left(2+\frac{1}{a_N}\right) - \frac{1}{2} \ln{2} + (\gamma_1 - \gamma_2) + o(1) \\
    \to & (\gamma_1 - \gamma_2)^* \quad \text{as } N \to \infty
\end{aligned}$$
* $\forall i \in \mathbb{N} ,\, p_i ,\, q_i \in \mathbb{N_+} \implies a_N \ge N \implies \lim_{N\to\infty} a_N = +\infty$

Denote $\gamma := \gamma_1 - \gamma_2$, then $s_N = \gamma + \frac{1}{2} \ln{\frac{p_1 + p_2 + \cdots + p_N}{q_1 + q_2 + \cdots + q_N}} + o(1)$ as $N\to\infty$.

\paragraph{(b)}
$\forall x>0$, denote $r:=2(x-\gamma)$.
$$e^r = \lim_{n\to\infty} \left( 1 + \frac{1}{n} \right) ^ {nr} = \lim_{n\to\infty} \frac{(1+n)^{nr}}{n^{nr}}$$
We let $p_1 = 2^{[r]} ,\, p_n = (1+n)^{[nr]} - n^{[(n-1)r]}$ and $q_1 = 1 ,\, q_n = n^{[nr]} - (n-1)^{[(n-1)r]}$,
\\\\
then $\forall i \in \mathbb{N} ,\, p_i ,\, q_i \in \mathbb{N_+}$, and that $\lim_{N\to\infty} \frac{p_1 + p_2 + \cdots + p_N}{q_1 + q_2 + \cdots + q_N} = \frac{(1+n)^{[nr]}}{n^{[nr]}} = \left( 1 + \frac{1}{n} \right) ^ {[nr]}$.
$$\begin{aligned}
    &\left\{ \begin{array}{l}    \left( 1 + \frac{1}{n} \right) ^ {nr-1} \le \left( 1 + \frac{1}{n} \right) ^ {[nr]} < \left( 1 + \frac{1}{n} \right) ^ {nr} \\
                                \lim_{n\to\infty} \left( 1 + \frac{1}{n} \right) ^ {nr-1} = \lim_{n\to\infty} \left( 1 + \frac{1}{n} \right) ^ {nr} = e^r \end{array} \right. \\
    \implies & \lim_{n\to\infty} \left( 1 + \frac{1}{n} \right) ^ {[nr]} = e^r \\
    \implies & \lim_{N\to\infty} \frac{1}{2} \ln \frac{p_1 + p_2 + \cdots + p_N}{q_1 + q_2 + \cdots + q_N} = x - \gamma\\
    \implies & \lim_{N\to\infty} s_N = x
\end{aligned}$$

\end{document}