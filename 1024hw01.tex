\documentclass{article}
\usepackage[fontsize=11pt]{fontsize}
\usepackage{graphicx,amsmath,amsfonts,mathabx, amsthm, amssymb, bm, hyperref, mathrsfs}
\usepackage{fontspec}
\usepackage{geometry}
\usepackage{boondox-cal}
\usepackage{color}
\geometry{a4paper,scale=0.75}
\setmainfont{Times New Roman}
\title{ MATH1024 Problem Set 1}
\author{ Group member:}
\date{February 2024}
\begin{document}
\setlength {\parindent} {0pt}
\maketitle







\paragraph{3.(a)Proof:}\ \\Firstly, we show that $\Omega_1 \cap \Omega_2$ is Jordan$\cdot$
measurable:\\We use Proposition 4.4. Given that $\Omega_1$ and $\Omega_2$ are Jordan measurable, there exist sequences
of inner simple regions $\{S_n^{(i)}\}^{\infty }_{n=1}$
and outer simple regions $\{T_n^{(i)}\}^\infty_{n=1}$, where i = 1, 2, such that
$\{S_n^{(i)}\}^{\infty }_{n=1}\subset \Omega_i \subset \{T_n^{(i)}\}^\infty_{n=1}$  for each $n \in N$ and  $i \in {1, 2}$, and
$$\lim_{n\to\infty}A(S_n^(i))=\lim_{n\to\infty}A(T_n^{(i)})=\mu(\Omega_i)$$
By induction, for two simple regions$X\subset Y$,
it's intuitive that $X\cap Y$,$X\cup Y$ and $X\slash Y$ are all simple regions.
From elementary set theory, we know 
$$(T_n^{(1)}\cap T_n^{(2)})\slash (S_n^{(1)}\cap S_n^{(2)}) \subseteq 
\left((T_n^{(1)}\cap T^{(2)}_2)\slash S_n^{(1)}\right)\cup\left(
    (T_n^{(1)}\cap T_n^{(2)})\slash S_n^{(2)}
\right)
\subseteq (T_n^{(1)}\slash S_n^{(1)})\cup (T_n^{(2)}\slash S_n^{(2)})$$
Therefore, we get:
$$A\left((T_n^{(1)}\cap T_n^{(2)})\slash (S_n^{(1)}\cap S_n^{(2)})\right)\leq
A\left((T_n^{(1)}\slash S_n^{(1)})\cup (T_n^{(2)}\slash S_n^{(2)})\right)\leq 
A(T_n^{(1)}\slash S_n^{(1)})+A(T_n^{(2)}\slash S_n^{(2)})
$$
which means that:
$$0\leq A(T_n^{(1)}\cap T_n^{(2)})-A (S_n^{(1)}\cap S_n^{(2)})\leq
 A(T_n^{(1)})+A(T_n^{(2)})-A(S_n^{(1)})-A(S_n^{(2)})$$
 For we have $$\lim_{n\to\infty}\left(A(T_n^{(1)})+A(T_n^{(2)})-
 A(S_n^{(1)})-A(S_n^{(2)})\right)=0$$,
 by squeeze theorem and Bolzano-Weierstrass Theorem, we can easily find that:
$$\lim_{n\to\infty}A(T_{n_j}^{(1)}\cap T_{n_j}^{(2)})=
\lim_{n\to\infty}A (S_{n_k}^{(1)}\cap S_{n_k}^{(2)})$$
We know that $A(T_{n_j}^{(1)}\cap T_{n_j}^{(2)})\supset \Omega_1\cap\Omega_2$ and 
$A(S_{n_k}^{(1)}\cap S_{n_k}^{(2)})\subset \Omega_1\cap\Omega_2$.\\
By Proposition 4.4, $\Omega_1\cap\Omega_2$ is Jordan measurable with
$$\mu (\Omega_1\cap \Omega_2)=\lim_{n\to\infty}A (S_{n_k}^{(1)}\cap S_{n_k}^{(2)})$$
\\ \\
Secondly, we will show that $\Omega_1\slash \Omega_2$ is Jordan measurable:\\
We have known that:
$$\Omega_1\slash \Omega_2\equiv \Omega_1\slash (\Omega_1\cap\Omega_2),(\Omega_1\cap\Omega_2)\subset\Omega_1$$
Let $\Omega_3\equiv (\Omega_1\cap\Omega_2)$.
Given that $\Omega_1$ and $\Omega_3$ are Jordan measurable, there exist sequences
of inner simple regions $\{S_n^{(i)}\}^{\infty }_{n=1}$
and outer simple regions $\{T_n^{(i)}\}^\infty_{n=1}$, where i = 1, 3(which is different from the above $T_n$ and $S_n$), such that
$\{S_n^{(i)}\}^{\infty }_{n=1}\subset \Omega_i \subset \{T_n^{(i)}\}^\infty_{n=1}$  
for each $n \in N$ and  $i \in \{1, 3\}$, and

$$\lim_{n\to\infty}A(S_n^(i))=\lim_{n\to\infty}
A(T_n^{(i)})=\mu(\Omega_i)$$

Then by $T_n^{(1)}\supset S_n^{(3)}$,  we can easily find that
$$ A(T_n^{(1)})-A(S_n^{(3)})= A(T_n^{(1)}\slash S_n^{(3)})\geq A(S_n^{(1)}\slash T_n^{(3)})\geq A(S_n^{(1)})-A(T_n^{(3)}) $$
As we know
$$\lim_{n\to\infty}A(T_n^{(1)})-A(S_n^{(3)})=\lim_{n\to\infty}A(S_n^{(1)})-A(T_n^{(3)})=\mu(\Omega_1)-\mu(\Omega_3)$$
then by squeeze theorem we have
$$\lim_{n\to\infty} A(T_n^{(1)}\slash S_n^{(3)})=\lim_{n\to\infty} A(S_n^{(1)}\slash T_n^{(3)})$$


$\because T_n^{(1)}\slash S_n^{(3)}\supset \Omega_1\slash\Omega_3$ and 
$S_n^{(1)}\slash T_n^{(3)}\subset \Omega_1\slash\Omega_3$\\
$\therefore $By Proposition 4.4,$\Omega_1\slash\Omega_3$ is Jordan measurable\\
Recall that $\Omega_3\equiv (\Omega_1\cap\Omega_2)$, 
we finally get $\Omega_1\slash\Omega_2$ is Jordan measurable.(Similarly,$\Omega_2\slash\Omega_1$ is Jordan measurable)



\paragraph{3.(b)Proof:}\ \\We have known that 
$$\Omega_1\cup \Omega_2\equiv (\Omega_1\slash \Omega_2)
\cup(\Omega_1\cap\Omega_2)\cup(\Omega_2\slash\Omega_1) $$
and by finite additivity we get:
$$\mu(\Omega_1)=\mu(\Omega_1\slash \Omega_2)+\mu(\Omega_1\cap\Omega_2)$$
$$\mu(\Omega_2)=\mu(\Omega_2\slash \Omega_1)+\mu(\Omega_1\cap\Omega_2)$$
$$\mu(\Omega_1\cup \Omega_2)=\mu(\Omega_1\slash \Omega_2)
+\mu(\Omega_1\cap\Omega_2)+\mu(\Omega_2\slash\Omega_1) $$
So finally we have:
$$\mu(\Omega_1\cup\Omega_2)=\mu(\Omega_1)+\mu(\Omega_2)-\mu(\Omega_1\cap\Omega_2)$$



\paragraph{3.(c)Proof:}\ \\Recalling the second part of 3(a), we have got:
$$ \lim_{n\to\infty}A(T_n^{(1)}\slash S_n^{3})
=\lim_{n\to\infty}A(S_n^{(1)}\slash T_n^{(3)})=\mu(\Omega_1\slash\Omega_3)$$
when $\Omega_3\subset\Omega_1$,which means that
$$\mu(\Omega_1\slash\Omega_3)=\mu(\Omega_1)-\mu(\Omega_3)$$
It's equivlent to show that if $\Omega_1\subset \Omega_2$, then 
$\mu(\Omega_2\slash\Omega_1)=\mu(\Omega_2)-\mu(\Omega_1)$




\paragraph{4.Proof} \ \\
Let $f(x)$ is a monotone bounded function on $[a, b]$,
 by Equation 4.1 and Isometric Invariance of Jordan measure, without loss of generality, we can 
 assume that $f(a)=0$ and $f(x)$ is monotone increasing on $[a,b]$, so
 $\forall x\in [a,b],f(x)\in [0,f(b)].$\\
 We will construct a sequence of patitions $P_n$ such that
 $U(P_n,f)$ and $L(P_n,f)$ converge to the same limit.\\
 Let $P_n$ be the partition $$
 x_0:=a<a+\frac{b-a}{n}<a+\frac{2(b-a)}{n}<.
 ..<a+\frac{(n-1)(b-a)}{n}<a+\frac{n(b-a)}{n}=b=:x_n$$
Then, for any $i\in \{1,...,n\}$ we have
$$M_i:= \sup_{x\in [x_{i-1},x_i]} f(x)=f(x_i)
$$
$$m_i:= \inf_{x\in [x_{i-1},x_i]} f(x)=f(x_{i-1})
$$
Hence,
$$
U(P_n,f)=\sum_{i=1}^{n}M_i\cdot \frac{b-a}{n}
=\sum_{i=1}^{n}f(x_i)\cdot\frac{b-a}{n}$$
$$
L(P_n,f)=\sum_{i=1}^{n}m_i\cdot \frac{b-a}{n}
=\sum_{i=1}^{n}f(x_{i-1})\cdot\frac{b-a}{n}$$
Firstly, we can easily find that:
$$U(P_n,f)-L(P_n,f)=(f(x_n)-f(x_0))\cdot\frac{b-a}{n}=\frac{b(b-a)}{n}$$
Let $n\to\infty$,then:
$$\lim_{n\to\infty}U(P_n,f)-L(P_n,f)=0$$
Next, we will show that a subsequence of $U(P_n,f)$ converges:

\begin{align*}
    U(P_{2^n},f)&=\sum_{i=1}^{2^n}M_i\cdot \frac{b-a}{2^n}
=\sum_{i=1}^{2^n}f(x_i)\cdot\frac{b-a}{2^n}\\
U(P_{2^{n+1}},f)&=\sum_{i=1}^{2^{n+1}}M_i\cdot \frac{b-a}{2^{n+1}}
\\&=\sum_{i=1}^{2^{n+1}}f(x_i)\cdot\frac{b-a}{2^{n+1}}\\
&= \sum_{i=1}^{2^{n+1}}f(x_i)\cdot\frac{b-a}{2^{n+1}}+
\sum_{i=1}^{2^{n+1}}f(x_{i-1})\cdot\frac{b-a}{2^{n+1}}
\end{align*}
Thus 
$$U(P_{2^{n+1}})-U(P_{2^n},f)=\sum_{i=1}^{2^{n+1}}(f(x_{i-1})-f(x_i))\cdot\frac{b-a}{2^{n+1}}<0$$
Therefore $U(P_{2^n},f)$ is a bounded and monotone decreasing sequence, the limit of 
$U(P_{2^n},f)$ which is a subsequence of $U(P_n,f)$) exists.
As we know:
$U(P_{2^n},f)-L(P_{2^n},f)$ is a subsequence of $U(P_n,f)-L(P_n,f)$,
namely
$$\lim_{n\to\infty}U(P_{2^n},f)=\lim_{n\to\infty}L(P_{2^n},f)$$
According to Proposition 4.7, $f$ is Riemann integrable.








 \end{document}
