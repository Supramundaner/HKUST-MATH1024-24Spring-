\documentclass{article}
\usepackage[fontsize=11pt]{fontsize}
\usepackage{CJK,indentfirst,graphicx,amsmath,amsfonts,mathabx, amsthm, amssymb, bm, hyperref, mathrsfs}
\usepackage{fontspec}
\usepackage{geometry}
\usepackage{boondox-cal}
\usepackage{color}
\geometry{a4paper,scale=0.75}
\setmainfont{Times New Roman}
\title{ MATH1024 Problem Set 3}
\author{ Group member:}
\date{February 2024}
\begin{document}
\setlength {\parindent} {0pt}
\maketitle

\paragraph{2.}
(a) Proof:\\
Suppose $u(x)=g(x)$, then $$\int_{a}^{b} g^2(x)dx=0$$. \\
According to Prop 4.12, as we have 
$$g^2(x)\geq 0,$$
$\therefore$ $g(x)=0$ for all $x\in[a,b]$\\
(b) Proof:\\
Firstly we justify that
$$\int_{a}^{b}(h(x)-\underline{h})dx=0.$$
Suppose $$H(x)=\int_{a}^{x}h(t)dt-\int_{a}^{x}\underline{h}dt,$$
then $$H(b)=\int_{a}^{b}(h(t)-\underline{h})dt=\int_{a}^{b}h(t)dt-\int_{a}^{b}\underline{h}dt.$$
Combining  with
$$\underline{h}=\frac{1}{b-a}\int_{a}^{b}h(t)dt,$$
we can find out that 
$$H(b)=\int_{a}^{b}h(t)dt-\int_{a}^{b}h(t)dt=\int_{a}^{b}(h(t)-\underline{h})dt=0.$$
Secondly we prove the $h$ is a constant function on $[a,b]$:
Suppose $v(x)=h(x)-\underline{h}$
\begin{align*}
    \int_{a}^{b}h(x)u(x)dx&=\int_{a}^{b}(v(x)+\underline{h})u(x)dx\\
    &=\int_{a}^{b}v(x)u(x)dx+\underline{h}\int_{a}^{b}u(x)dx\\
    &=\int_{a}^{b}v(x)u(x)dx
\end{align*}
According to the previous deduction, we have
$$\int_{a}^{b}v(x)dx=0$$
Then let $u(x)=v(x)$, then
$$\int_{a}^{b}v(x)u(x)=\int_{a}^{b}v^2(x)dx=0$$
According to Prop 4.12, $v^2(X)=0$ for all $x\in [a,b]$, namely 
$v(x)=0$. Recalling that $v(x)=h(x)-\underline{h}$, we have
$$h(x)=\underline{h}$$
Thus $h(x)$ is a constant function.

\paragraph[short]{3.}
Proof: \\
(ii)$\implies$(i):
\begin{align*}
    y(t)&=\cos t+\int_{0}^{t}s^2\sin(s-t)y(s)ds\\
    &=\cos t +\cos t\int_{0}^{t}s^2\sin s y(s)ds -\sin t\int_{0}^{t}
    s^2\cos sy(s)ds
\end{align*}
So by differentiating $y(t)$, we can get:
\begin{align*}
    y'(t)&=-\sin t +\left(-\sin t \int_{0}^{t}s^2\sin s\cdot y(s)ds+
    \sin t \cos t t^2 y(t)\right)-\left(\cos t \int_{0}^{t}
    s^2 \cos s\cdot y(s)ds+\sin t \cos t t^2 y(t)\right)\\
    &=-\sin t -\sin t \int_{0}^{t}s^2\sin s\cdot y(s)ds-
    \cos t \int_{0}^{t}s^2 \cos s \cdot y(s)ds
\end{align*}
Furthermore, we can differentiate $y'(t)$ and get second derivative of $y(t)$:
\begin{align*}
    y''(t)&=-\cos t-\left(\cos t\int_{0}^{t}s^2\sin s \cdot y(s)ds+
    \sin^2 t\cdot t^2y(t) \right)-\left(\sin t\int_{0}^{t}s^2\cos s\cdot y(s)ds+\cos^2 \cdot t^2y(t) \right)\\
    &=-\left(\cos t +\cos t\int_{0}^{t}s^2\sin s \cdot y(s)ds-\sin t\int_{0}^{t}s^2\cos s\cdot y(s)ds\right)-\left(
    (\sin^2 t+\cos^2 t)t^2y(t) \right)
\end{align*}
Recalling that 
$$ y(t)=\cos t +\cos t\int_{0}^{t}s^2\sin s y(s)ds -\sin t\int_{0}^{t}
s^2\cos sy(s)ds$$
$$\sin^2 t+\cos^2 t=1$$
We can rewrite $y''(t)$ as 
$$y''(t)=-y(t)-t^2y(t)$$
Also it's trivial to find that $y(0)=1,y'(0)=0$. Thus:
$$y''(t)+(1+t^2)y(t)=0 \forall t\in\mathbb{R}, y(0)=1, y'(0)=0$$
(i)$\implies$(ii):\\
Denote
$$f(t)=\cos t\int_{0}^{t}s^2\sin s y(s)ds 
-\sin t\int_{0}^{t}s^2\cos sy(s)ds$$
Then it's similar to the calculation on the above to get:
$$f'(t)=-\sin t \int_{0}^{t}s^2\sin s\cdot y(s)ds-
\cos t \int_{0}^{t}s^2 \cos s \cdot y(s)ds$$
$$f''(t)=-\left(\cos t\int_{0}^{t}s^2\sin s \cdot y(s)ds-\sin t\int_{0}^{t}s^2\cos s\cdot y(s)ds\right)-
    t^2y(t)$$
Then rearrange the functions we have, we can easily find
$$\left\{
    \begin{aligned}
        f''(t)=-f(t)-t^2y(t)\\
        y''(t)=-y(t)-t^2y(t)
    \end{aligned}
\right.
$$
So we can find out
$$f''(t)-y''(t)=-\left(f(t)-y(t)\right)$$
Denote $\varphi (t)=f(t)-y(t)$, then $\varphi ''(x)=-\varphi (x)$.\\
Let $g(t)=\frac{\varphi (t)}{\cos t} (t\neq\frac{\pi}{2}+k\pi)$,
then differentiate $g(t)$, we can get:
$$g'(t)=\frac{\varphi '(x)\cos t+\sin t\cdot \varphi (t)}{\cos^2 t}$$
Denote $\lambda (x)=\varphi '(x)\cos t+\sin t \cdot \varphi (t)$, then
differentiate $\lambda(t)$, we can get:
\begin{align*}
    \lambda'(t)&=\varphi ''(t)\cos t-\varphi '(t)\sin t+\varphi '(t)\sin t 
    +\cos t\varphi(t)\\
    &=\cos t(\varphi(x)+\varphi ''(x))\\
    &=0
\end{align*}
Thus $\lambda(X)$ is a constant function.\\
Recalling that $f(0)=0,y(0)=1,f'(0)=0,y'(0)=0$, we have
$\varphi(0)=-1,\varphi'(0)=0$. \\
Thus $\lambda(0)=0$, which means
$g'(t)=0$.\\
$\therefore g(t)=\frac{\varphi (t)}{\cos t}$ is a constant function.\\
$\because \varphi(0)=-1,\cos 0=1$\\
$\therefore g(0)=-1,g(t)=\frac{\varphi (t)}{\cos t}=-1((t\neq\frac{\pi}{2}+k\pi))$
$\therefore \varphi (t)=-\cos t(t\neq\frac{\pi}{2}+k\pi)$\\
For $t=\frac{\pi}{2}+k\pi(k\in\mathbb{Z})$, by the continuity of $\varphi (t)$, we can still get
$$\varphi (t)=-\cos t$$
Combining with $\varphi (t)=f(t)-y(t)$,
\begin{align*}
    y(t)&=f(t)-\varphi(t)\\
    &=\cos t +\cos t\int_{0}^{t}s^2\sin s y(s)ds -\sin t\int_{0}^{t}
    s^2\cos sy(s)ds\\
    &=\cos t+\int_{0}^{t}s^2\sin(s-t)y(s)ds, \forall t\in\mathbb{R}
\end{align*}






\end{document}